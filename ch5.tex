\chapter{Results and Discussion} \label{cha:results}
A crucial component of any modeling effort is to confirm
that the model agrees with experimental data to a level
appropriate for the intended application.
This validation process confirms the appropriateness of the simplifying
assumptions made during the model development.
For the \IP{1500}{\horsepower} gas test block, experimental data are available for 
numerous tests conducted in July 2005 and November 2012.
The approach used for these tests is described in \cref{sec:approach}.

The complete cycle model
was run with inputs matching each experimental data set
to validate the model over a wide range of operating conditions.
Because outdoor air conditions were not recorded at the time of the test,
the conditions have been sourced from weather data archives \parencite{wunder2005,wunder2012}.
The resulting model output variables are compared with the
experimental data in \cref{sec:expresults}.
Parity plots are used to illustrate the agreement 
between the model results and experimental data.
The results agree to within xx\% in the compressor and orifice modules,
and to within yy\% in the condenser and cooling tower modules,
which required broader assumptions.

\Cref{sec:sensitivity} studies the effect of intermediate
variables on the overall results.
For example, subtle differences between the calculation methods used
in the model and used during experiments mean that the Reynolds numbers
for the compressor and the orifice
do not agree with the experimental data nearly as well
as other variables.
However, while the Reynolds numbers have a subtle effect on the 
compressor discharge state and the orifice
differential pressure, the sensitivity analysis
reveals that the overall results are not significantly affected.
Therefore, the sensitivity analysis is an essential step in
proving the robustness of the model.

\section{Experimental Approach} \label{sec:approach}
The test data from JCI were obtained using the guidelines
in \textcite{bluebook} and in accordance with \textcite{ptc10},
as described in \cref{sec:TestMethodology}.
A subset of the test data used for validation purposes
is shown in \cref{tab:SampleData} of \cref{app:sample}.
Proprietary JCI compressor design information has 
been removed from the sample data.

\section{Comparison of Experimental and Model Results} \label{sec:expresults}


\section{Sensitivity Analysis} \label{sec:sensitivity}

