\chapter{Solution Methodology} \label{cha:solnmethod}

The equations developed in \cref{cha:modeldev} are implemented in
Engineering Equation Solver (EES). 
This software package provides thermophysical property data, unit checking, 
a graphical user interface (GUI) framework, and an automatic equation 
blocking scheme for iteratively solving systems of simultaneous equations. 
These features, along with compressor engineers' familiarity with the program, 
make EES an ideal choice for implementation of the gas test block model.

The model is implemented using an EES module for each device in the cycle. 
As much as possible, the equations and assumptions for each device are contained
within the corresponding module.
This facilitates a semantic calling structure in the main body of the program 
and will simplify code maintenance for future modifications or improvements 
to the model.
The overall structure of the implementation is described in \cref{sec:Overall}.

One of the goals of the project, the ability to select the best flow measurement
orifice for a user-input range of operating points, requires multiple operating points
to be evaluated.
The remainder of the goals consist of calculations for a single operating point at a time.
A limitation of the EES software restricts the number of variables stored.
Therefore, it is not feasible to store the results for the entire cycle at multiple operating points.
For this reason, two EES programs were developed, each sharing a common code base. 

The first program computes results for the entire cycle, but only 
for a single operating point at a time.
This program is used for all use cases except orifice selection
and is discussed in \cref{sec:MainModel}.
The second program has the capability to evaluate and store orifice and 
compressor model results for multiple operating points.
It calls only the compressor and orifice modules of the code to minimize the number of variables required.
This model is used for orifice selection and is discussed in \cref{sec:OrifSelect}.

\section{Structure} \label{sec:Overall}

\section{Complete Cycle Program} \label{sec:MainModel}

\section{Orifice Selection Program} \label{sec:OrifSelect}

\begin{figure}
  \centering
  \includegraphics{figs/OrificeSelectionFlowchart}
  \caption{}
  \label{fig:FlowOrifSelect}
\end{figure}