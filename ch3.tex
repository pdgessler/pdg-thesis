\chapter{Theoretical Model Development} \label{cha:modeldev}
\note{This is bare-bones right now. Please check style and level of detail.}
\todo{Add references throughout!}
This chapter presents the development of the equations used to model 
the compressor test block from first principles. 
\Cref{sec:GenBal} contains the general mass and energy balance equations, 
to which the appropriate assumptions for each device in the cycle are applied 
to produce the device-specific equations, as detailed in \crefrange{sec:Compressor}{sec:Mixing}. 
Subscripts $\text{i}$ and $\text{o}$ are used to indicate 
the inlet and outlet states of the device, respectively. 

\section{General Balance Equations} \label{sec:GenBal}
Throughout the model, the rate forms of the balance equations are used 
since compressor test data from the block are recorded once steady-flow conditions are reached. 
This section lists the mass and energy balances in their most general form, 
so that engineering assumptions may be applied on a per-device basis.

The general mass balance is obtained by 
applying conservation of mass to the control volume. 
This process yields
\begin{equation}
	\dod{m_\CV}{t} = \sum_{\text{inlets}}{\dot{m}} - \sum_{\text{outlets}}{\dot{m}},
  \label{eqn:GenMass}
	\nomenclature{$t$}{Time\nomunit{\minute}} \nomenclature{$m$}{Mass\nomunit{\kilogram}} 
	\nomenclature[X]{$s$}{Isentropic} \nomenclature[X]{$\CV$}{Control volume}
\end{equation}
where the summations account for any number of inlets and outlets to the device.
\Cref{eqn:GenMass} includes only transport terms since mass production is 
always identically zero (neglecting nuclear processes).

Similarly, the general energy balance is obtained by writing 
an extensive property balance over the control volume. 
By the first law of thermodynamics, energy production is identically zero, 
so again, only transport terms remain in the equation. 
Here though, heat transfer and work transfer interactions 
can change the energy of the control volume, so
\begin{equation}
	\dod{E_\CV}{t} = \dot{Q} - \dot{W} + \sum_{\text{inlets}}{\dot{m}h^{\circ}} - 
		\sum_{\text{outlets}}{\dot{m}h^{\circ}},
  \label{eqn:GenEnergy}
	\nomenclature{$e$}{Specific energy, $e=u+\text{ke}+\text{pe}$\nomunit{\kilo\joule\per\kilogram}} 
	\nomenclature{$Q$}{Heat transfer\nomunit{\kilo\joule}} 
	\nomenclature{$W$}{Work transfer\nomunit{\kilo\joule}} 
	\nomenclature[Y]{$\dot{\left(\;\right)}$}{Quantity per unit time or rate, $\dif{\left(\;\right)}/\mkern-3mu\dif{t}$\nomunit{$\left[\;\right]$\per\second}}
	\nomenclature{$h^{\circ}$}{Specific methalpy, $h^\circ=h+\text{ke}+\text{pe}$\nomunit{\kilo\joule\per\kilogram}} 
	\nomenclature[A$k$]{$\text{ke}$}{Specific kinetic energy, $\text{ke}=\vel^2/2$\nomunit{\kilo\joule\per\kilogram}}
	\nomenclature[A$p$]{$\text{pe}$}{Specific potential energy, $\text{pe}=gz$\nomunit{\kilo\joule\per\kilogram}}
	\nomenclature{$z$}{Elevation\nomunit{\metre}}
	\nomenclature{$g$}{Gravitational acceleration, $g=\SI{9.81}{\metre\per\second\squared}$\nomunit{\metre\per\second\squared}}
	\nomenclature{$h$}{Specific enthalpy, $h=u+pv$\nomunit{\kilo\joule\per\kilogram}}
	\nomenclature{$p$}{Pressure\nomunit{\kilo\pascal}}
	\nomenclature{$v$}{Specific volume\nomunit{\metre\cubed\per\kilogram}}
	\nomenclature[W]{$\rho$}{Density, $1/v$\nomunit{\kilogram\per\metre\cubed}}
	\nomenclature{$u$}{Specific internal energy\nomunit{\kilo\joule\per\kilogram}}
  \nomenclature[A$z$]{*}{Unless otherwise noted, symbols for extensive properties are the uppercase variants of symbols used for the corresponding specific (intensive) properties.}
\end{equation}
where the \emph{methalpy}, $h^\circ = h+\text{ke}+\text{pe}$, 
is a convenient way to account for kinetic and potential energy effects if required. 
The methalpy can be calculated from the enthalpy, the kinetic energy
\begin{equation}
  \text{ke} = \vel^2/2,
  \nomenclature{$\vel$}{Velocity magnitude\nomunit{\metre\per\second}}
\end{equation}
where the average velocity $\vel$ can be calculated using 
$\dot{m}=\rho \vel A$, and the potential energy
\begin{equation}
  \text{pe} = gz,
\end{equation}
where $g$ is the gravitational acceleration 
and $z$ is the elevation of the inlet/outlet.

\Cref{eqn:GenEnergy} shows the standard sign convention used throughout this thesis; 
that is, heat transfer \emph{into} the system is considered positive, 
while work transfer \emph{out of} the system is considered positive.
\Cref{eqn:GenMass,eqn:GenEnergy} form the basic equations to which 
engineering assumptions are applied in \crefrange{sec:Compressor}{sec:Mixing}.
Assumptions common to each device in the cycle are that
\begin{enumerate}
  \item the steady-flow condition leads to both time derivatives $\dif{m_\CV}/\!\dif{t} = \dif{E_\CV}/\!\dif{t} = 0$, \label{lst:sf}
  \item changes in potential energy are neglected, so $\Delta \text{pe} = 0$, and \label{lst:pe}
  \todo{generalize for MISO or SIMO devices.}
  \item changes in kinetic energy are neglected, so $\Delta \text{ke} = 0$. \label{lst:ke}
\end{enumerate}
As a consequence of assumptions~\ref{lst:pe} and~\ref{lst:ke},
we can reduce the methalpy notation to the more familiar enthalpy form,
and assumption~\ref{lst:sf} means that the left-hand sides of \cref{eqn:GenMass,eqn:GenEnergy} both become zero:
\begin{align}
  0 &= \sum_{\text{inlets}}{\dot{m}} - \sum_{\text{outlets}}{\dot{m}}, \label{eqn:SemiMass}\\
  0 &= \dot{Q} - \dot{W} + \sum_{\text{inlets}}{\dot{m}h} - \sum_{\text{outlets}}{\dot{m}h}. \label{eqn:SemiEnergy}
\end{align}

\section{Compressor} \label{sec:Compressor}
Detailed modeling of the compressor is complex and worthy of a dissertation in its own right. 
To make matters worse, the model should make reasonable predictions 
of a compressor test for \emph{any} compressor, 
whether one existing now or one to be developed in the future.
\note{Parallel structure seems odd on this sentence.} 
Fortunately, detailed compressor models are not required 
to predict compressor performance for the purposes of the present facility modeling. 
The performance of the compressor is provided to the model through compressor maps. 
An example of a generic compressor map is shown in \cref{fig:GenMap}. 
\begin{figure}[htbp]
  \centering
  %\includegraphics{figs/GenericCompMap}
  PLACEHOLDER FOR GENERIC (CONTRIVED) COMP.~MAP
  \caption{Generic compressor map, showing flow and head axes and efficiency islands.}
  \label{fig:GenMap}
\end{figure}
These compressor maps are developed by JCI either experimentally 
for existing compressors in the present facility or numerically%
---using other compressor design tools, from basic one-dimensional methodologies
to full three-dimensional CFD---for new compressor designs.

A user of the model can select an operating point on the compressor map, 
and with specification of the suction (inlet) conditions, 
the discharge (outlet) state of the compressor is fixed. 
Dimensionless coefficients of flow, head, and velocity
($\Theta$, $\Omega$, and $\MA$, respectively) are used 
to generalize the compressor maps to machine characteristics. 
The flow coefficient $\Theta$ is given by
\begin{equation}
  \Theta = \frac{\dot{V}}{\svel D^2},
	\nomenclature{$\svel$}{Acoustic (sonic) velocity\nomunit{\metre\per\second}}
	\nomenclature[W]{$\Theta$}{Flow coefficient, $\dot{V}/{\svel D^2}$\nomunit{-}}
\end{equation}
where $\dot{V}$ is the volumetric flow, 
$\svel$ is the acoustic (sonic) velocity at suction conditions, 
and $D$ is the impeller tip diameter.
The head coefficient $\Omega$ is given by
\begin{equation}
	\Omega = \frac{g_c \Delta h_s}{\svel^2},
	\nomenclature[W]{$\Omega$}{Isentropic head coefficient, ${g_c \Delta h_s}/{\svel^2}$\nomunit{-}}
	\nomenclature{$g_c$}{Gravitational proportionality constant, $g_c=\SI{1}{\kilogram\metre\per\newton\second\squared}$\nomunit{\kilogram\metre\per\newton\second\squared}}
\end{equation}
where $g_c$ is the gravitational proportionality constant, 
$\Delta h_s$ is the specific enthalpy change for an isentropic compression process, 
and $\svel$ is, again, the suction sonic velocity.

The Machine Mach number $\MA$ is 
\begin{equation}
  \MA = \frac{\vel_{\text{tip}}}{a},
  \nomenclature[A$Ma$]{$\MA$}{Machine Mach number, $\vel_{\text{tip}}/a$\nomunit{-}}
\end{equation}
where $\vel_{\text{tip}}$ is the impeller tip velocity
and $a$ is the acoustic velocity at inlet (suction) conditions.
The third and final compressor performance characteristic gleaned 
from the compressor map is the isentropic efficiency, $\eta_s$, 
which quantifies the deviation of the real compression process 
from an isentropic compression process. 
The isentropic efficiency is defined as
\begin{equation}
  \eta_s = \frac{h_{\text{o}s}-h_{\text{i}}}{h_{\text{o}}-h_{\text{i}}},
\end{equation}
where $h_{\text{o}s}$ is the discharge enthalpy for an isentropic compression process, 
$h_{\text{i}}$ is the suction enthalpy, 
and $h_{\text{o}}$ is the actual discharge enthalpy. 

While the model uses isentropic efficiency $\eta_s$, 
the efficiency islands on standard compressor maps plot a corrected efficiency $\eta_{\text{map}}$,
which includes a correction for the Reynolds number at the given operating point.
To convert the corrected map efficiency into isentropic efficiency, 
we use the transformation
\begin{equation}
  \eta_s = \eta_{\text{map}} + X(1 - \eta_{\text{peak}})\left(1-\left(\frac{10^6}{\RE_b}\right)^{0.1}\right),
\end{equation}
where the flow ratio $X = \max{(1,\Theta/\Theta_{\text{peak}})}$, 
$\Theta_{\text{peak}}$ is the flow coefficient 
corresponding to the point of maximum map efficiency,
$\eta_{\text{peak}}$ is the maximum map efficiency,
and $\RE_b$ is the Reynolds number 
based on impeller tip width $b$%
\nomenclature{$b$}{Impeller tip width\nomunit{\inch}}; for example,
\begin{equation}
\text{Re}_b=\frac{\rho \vel b}{\mu}.
\end{equation}

The four map parameters for the desired operating point 
($\Theta$, $\Omega$, $\MA$, and $\eta_s$), 
specified suction conditions ($T_{\text{i}}$ and $p_{\text{i}}$), 
and machine characteristics ($D$, $b$, $\Theta_{\text{peak}}$, and $\eta_{\text{peak}}$) 
allow the discharge conditions ($T_{\text{o}}$ and $p_{\text{o}}$) 
and mass flow rate $\dot{m}$ to be computed by the model. 
\todo{Flowchart? Maybe this should be left for Ch. 4.} 
This calculation process is outlined in \cref{fig:DischargeCalc}.
\begin{figure}[htbp]
  \centering
  PLACEHOLDER FOR DISCHARGE CONDITION CALCULATION FLOWCHART
  \caption{Flowchart of the process for calculating compressor discharge conditions.}
  \label{fig:DischargeCalc}
\end{figure}

Once the inlet and outlet states at the compressor are known, 
the mass and energy balances can be revisited. 
From \cref{eqn:SemiMass,eqn:SemiEnergy} and
 assuming that the compressor can be modeled as a 
single-input, single-output (SISO) device, we have
\begin{align}
  &\text{Mass:}   & 0 &= {\dot{m}_{\text{i}}} - {\dot{m}_{\text{o}}} \label{eqn:CompMass}\\
  &\text{Energy:} & 0 &= \dot{Q} - \dot{W} + {\dot{m}_{\text{i}}h_{\text{i}}} - 
		{\dot{m}_{\text{o}}h_{\text{o}}} \label{eqn:CompEnergy}.
\end{align}
\note{Using `we' is a habit of mine in derivations. OK or should I lose it?}
\Cref{eqn:CompMass} simply means that the mass flow 
entering the compressor is equal to that leaving the compressor. 
In \cref{eqn:CompEnergy}, we know the mass flow rates. 
Finally, assuming that the compressor is well-insulated; 
that is, $\dot{Q}=0$, leaves $\dot{W}$ 
as the sole remaining unknown in \cref{eqn:CompEnergy}. 
This value is one of the requirements of the model. 
\todo{Add mechanical efficiency stuff.}

\section{Orifice Flow Meter} \label{sec:Orifice}
\emph{PDG TODO: This section needs cleanup, text added, and equations (subscripts) fixed for consistency.}
\begin{equation}
  \dot{V} = \dot{m}_{\text{i}}v_{\text{i}}
\end{equation}
\begin{align}
  A_0 &= \frac{\pi}{4} d^2 \\
  A_1 &= \frac{\pi}{4} D_{\text{pipe}}^2 \\
  \vel &= \dot{V}/A_1 \\
  \RE_{D,\text{ pipe}} &= \frac{\rho_{\text{i}} \vel D_{\text{pipe}}}{\mu_{\text{i}}} \\
  \beta &= \frac{d}{D_{\text{pipe}}}
\end{align}
\begin{multline}
  C = 0.5959 + 0.0312\beta^{2.1} - 0.1840\beta^8 + 0.0029\beta^{2.5}\left(\frac{10^6}{\RE_d}\right)^{0.75} \\ 
      + 0.0900\left(\frac{L_1}{D_{\text{pipe}}}\right)\frac{\beta^4}{\left(1-\beta^4\right)} 
      - 0.0337\left(\frac{L_2}{D_{\text{pipe}}}\right)\beta^3.
\end{multline}
\begin{equation}
  \gamma = \frac{c_p}{c_v}
\end{equation}
\begin{equation}
  r = \frac{p_3}{p_2}
\end{equation}
\begin{equation}
  Y = \sqrt{r^{2/\gamma}\left(\frac{\gamma}{\gamma-1}\right)\frac{1-r^{(\gamma-1)/\gamma}}{1-r}\frac{1-\beta^4}{1-\beta^4r^{2/\gamma}}}
\end{equation}
\begin{equation}
  \dot{V} = C Y A_0 \sqrt{2v_2g_c\left(\frac{p_2-p_3}{1-\beta^4}\right)}
\end{equation}
\nomenclature[A$Re$]{$\RE_L$}{Reynolds number based on characteristic length $L$, $\text{Re}_L=\rho \vel L/\mu$\nomunit{-}}
\nomenclature[W]{$\mu$}{Dynamic viscosity\nomunit{\newton\second\per\metre\squared}}
\nomenclature{$D$}{Diameter\nomunit{\metre}}
\nomenclature{$d$}{Orifice diameter\nomunit{\metre}}
\nomenclature{$A$}{Area\nomunit{\metre\squared}}
\nomenclature[W]{$\beta$}{Ratio of orifice diameter to pipe diameter, $\beta = d/D$\nomunit{-}}
\nomenclature{$C$}{Orifice coefficient of discharge\nomunit{-}}
\nomenclature{$c_p$, $c_v$}{Specific heats, constant pressure and volume, respectively\nomunit{\kilo\joule\per\kilogram\kelvin}}
\nomenclature[W]{$\gamma$}{Ratio of specific heats, $\gamma = c_p/c_v$\nomunit{-}}
\nomenclature{$r$}{Orifice pressure ratio, $r=p_{\text{o}}/p_{\text{i}}$\nomunit{-}}
\nomenclature{$Y$}{Expansion factor\nomunit{-}}
\nomenclature[W]{$\eta$}{Efficiency\nomunit{-}}
\nomenclature{$x$}{Quality\nomunit{-}}
\nomenclature[X]{$\text{i}$}{Inlet conditions}
\nomenclature[X]{$i$}{The $i$th component/element of a mixture/series}
\nomenclature[X]{$\text{o}$}{Outlet conditions}
\nomenclature[Y]{$\Delta\left(\;\right)$}{Change in quantity; for example, $\text{final} - \text{initial or outlet} - \text{inlet}$\nomunit{$\left[\;\right]$}}
\nomenclature[X]{$\text{sh}$}{Superheat}
\nomenclature[0]{\textbf{Symbol}}{\textbf{Description}\nomunit{\textbf{Units}}}

FROM PTC 19.5:

\begin{equation}
C = K/E = K\sqrt{1-\beta^4}
\end{equation}
\begin{equation}
K = K_0 \left(1+\frac{A}{\RE_d}\right)
\end{equation}
\begin{equation}
K_0=K_e \left(\frac{10^6 d}{10^6 d + 15A}\right)
\end{equation}
\begin{multline}
K_e = 0.5993 + \frac{0.007}{D} + \left(0.364 + \frac{0.076}{\sqrt{D}}\right)\beta^4 
      + 0.4\left(1.6 - \frac{1}{D}\right)^5 \left[\left(0.07 + \frac{0.5}{D}\right) - \beta\right]^{5/2} \\
      - \left(0.009 + \frac{0.034}{D}\right) \left(0.5 - \beta\right)^{3/2} 
      + \left(\frac{65}{D^2} + 3\right)\left(\beta - 0.7\right)^{5/2}
\end{multline}
\begin{equation}
A = d\left(830 - 5000\beta + 9000\beta^2 - 4200\beta^3 + \frac{530}{\sqrt{D}}\right)
\end{equation}

\section{Flow Split} \label{sec:FlowSplit}

\section{Condenser} \label{sec:Condenser}

\section{Cooling Tower} \label{sec:CoolingTower}
\begin{align*}
  c = 1.684; &\quad n = - 0.391 \\
  \text{NTU} &= c\left(\frac{\dot{m}_{w,i}}{\dot{m}_{a,i}}\right)^{1 + n}
\end{align*}

\section{Expansion Device---Saturated Liquid-Vapor Mixture Stream} \label{sec:ExpandMain}

\section{Expansion Device---Superheated Stream} \label{sec:ExpandSHV}

\section{Mixing Chamber} \label{sec:Mixing}

