\chapter{Conclusions and Recommendations} \label{cha:conclusion}
This chapter contains a summary of the research work presented in this thesis
and discusses the resulting conclusions.
In addition, it contains recommendations for future improvements to the model
and test block data acquisition.
These data acquisition improvements would enable more extensive validation of the model, 
which in turn would improve the accuracy of the model by guiding the selection
of empirical parameters.

The motivations and requirements described in \cref{cha:intro} guided the development
of a thermodynamic model of the \IP{1500}{\horsepower} hot gas bypass test block
facility located in York, PA.
After presenting prior research in \cref{cha:litreview}, the simplifying assumptions 
used in developing the model were outlined in \cref{cha:modeldev}.
These simplified equations were numerically solved by implementation
in EES as discussed in \cref{cha:solnmethod}.
Graphical interfaces were presented for the two main use cases of the model:
the complete cycle simulation and the orifice selection program.

\Cref{cha:results} showed the results of the validation effort
in which model outputs were compared to experimental data from the test block.
These results showed that the model is accurate enough for the intended application
and that the assumptions applied in developing the equations are appropriate.
Sensitivity analyses conducted on several parameters in the model showed that
even when intermediate results do not fully agree with experimental data,
the primary outputs remain sufficiently accurate.

\section{Conclusions} \label{sec:conclusions}
Based on the results presented and discussed in \cref{cha:results},
the model meets the two primary goals of predicting whether test conditions
are feasible and assisting in orifice selection.
These goals were accomplished by predicting the air mass flow rate in the
cooling tower and predicting the differential pressure across the orifice.
If the predicted air mass flow rate exceeds the capacity of the cooling towers,
then it is likely that the specified test condition is infeasible at the given
outdoor air conditions.
Similarly, the orifice differential pressure must fall within a specified range
in order to be suitable for measuring the refrigerant mass flow rate.
Therefore, the orifice selection program allows engineers to specify
a desired range of test conditions and
then displays the orifice differential pressure for 
a number of operating points within the specified range.
This allows the engineer to make an informed decision about the best orifice 
for the test plan without tedious calculations.

For the validation cases of \cref{cha:results}, the maximum percent difference between model
and experimental results is 2.5\%, indicating that the model is quite accurate given the number of assumptions involved.
The model's prediction of cooling tower air flow rate cannot be assessed because
it is not measured during tests, but the primary output variable for 
orifice selection, the orifice differential pressure,
agrees with the experimental results to within 0.35\%.
It should be noted, however, that the validation encompassed only a
small range of test points and weather conditions, from which several
of the parameters in the model are defined.
Results for operating conditions far removed from the validation data
may vary.
Methods for correcting this shortcoming of the model are discussed in \cref{sec:recommend}.

\section{Recommendations} \label{sec:recommend}
As noted in \cref{cha:results} and \cref{sec:conclusions},
the condenser and cooling tower modules of the model make significant
assumptions to determine the parameters that characterize their performance.
These parameters include the overall heat transfer coefficient for the condenser, $UA$,
and the two characteristic parameters for the cooling tower model's 
calculation of $\NTU$, $c$ and $n$.
Calculating these parameters using first principles was not a central focus of this project, 
because they are affected by many complex transport processes.
Instead, these parameters were defined empirically, using average values calculated over the 
validation data sets.
For this reason, the model results may not be as accurate for test conditions
far removed from the validation test conditions.
To improve these portions of the model, a number of data collection
recommendations are prudent.

First, the cooling tower fan speeds should be recorded as part of each test.
Currently, the fan speeds are adjusted until the test block reaches steady state
at the desired operating point.
Combined with fan curves for the cooling towers installed on the block,
records of the fan speeds will allow calculation of the cooling tower
model parameters $c$ and $n$. 
This will enhance the accuracy of the cooling tower model and extend the range
of test conditions over which its accuracy is acceptable.
At the present time, the model uses typical values for these parameters from \textcite{braun1989}.

Similarly, a means to determine the condenser water flow rate would
greatly improve the accuracy of calculations for the condenser overall heat transfer coefficient, $UA$.
As discussed in \cref{sec:sensitivity}, the condenser water pump drives the
flow through both a bypass loop and the condenser loop. With the current instrumentation
setup, the division of flow between these loops is unknown.
If the condenser water flow rate is known, then a more reliable
estimate of $UA$ could be produced.

Other potential improvements to the model concern the adiabatic and constant pressure assumptions.
It is shown in \cref{sec:condresult} that the constant pressure assumption is valid for the sizes
of pipe used on the gas test block.
Additionally, the adiabatic assumption is shown to be valid for the orifice in \cref{sec:OrifResults}.
However, the adiabatic assumption for the compressor results in underprediction of the compressor
gas horsepower, as discussed in \cref{sec:CompResults}.
A model of the heat transfer from the compressor casing to the ambient conditions is likely to
improve this prediction.
Only the compressor module and the overall energy balance 
for the cycle would need to be modified to implement this improvement.

Because of the modular construction of the model, it will be relatively 
straightforward to implement any of these improvements.
The data collection additions on the gas test block may prove more difficult to complete,
but if these are possible, they would expand the validation possibilities 
and allow for better estimation of the characteristic parameters in the model.
In addition, the model serves its intended purpose as it currently stands.
Therefore, it is expected that the thermodynamic model will serve as a useful
tool for conducting efficient compressor tests at the \IP{1500}{\horsepower} gas
test block facility.