\chapter{Literature Review} \label{cha:litreview}

\section{General Compressor Testing} \label{sec:gencomp}
One component of the research effort was 
to study different methods of testing compressors. 
Therefore, this section provides information regarding performance metrics 
and testing methods for several types of compressors
commonly used in the HVAC industry. 
The research was done to better understand the testing methodology, 
the data collected during a typical test, 
and the expectations for a mathematical model of the test block refrigeration cycle.

\subsection{Compressor Testing Methodology} \label{sec:TestMethodology}
Two of the main standards organizations (the American Society of Mechanical Engineers (ASME) 
and the International Organization for Standardization (ISO)) 
publish guidelines for refrigerant compressor testing methods, data analysis, and results reporting.

The AMSE PTC 10 standard \citep{ptc10} prescribes test conditions, 
procedures, and measurement locations for a compressor test. 
This allows different manufacturers' test results to be compared. 
It also presents dimensionless coefficients used by different industries 
to characterize the operating point of the compressor. 
Finally, ASME PTC 10 contains several sets of sample calculations 
which are of use in verifying the implementation of the model, 
particularly the calculation of the prime mover power requirements.

\subsection{Compressor Performance Metrics}
It is important to understand the specific quantities used 
to describe a compressor's performance within the HVAC\&R%
\nomenclature[Z]{HVAC}{Heating, Ventilating, and Air Conditioning} 
industry, since these will be among the outputs of any model of the testing process. 
\citet{wilcox2007} advocates the use of ASME PTC-10 (\cref{sec:TestMethodology}) and lists
%\begin{itemize}
  suction and discharge pressures,
  suction and discharge temperatures,
  mass flow rate,
  fluid (refrigerant) composition,
  rotational speed, and
  driver load
%\end{itemize}
as critical field data for any compressor test.

Furthermore, Wilcox outlines some general guidelines 
for instrumentation on the test block to ensure representative data.  
He notes that pressure and temperature sensors should be located 
at least 10 pipe diameters from potential disturbances or obstructions such as tees or elbows.  
All sensors should be calibrated prior to the test run, 
and data should only be collected at steady-state conditions.  
For a typical compressor test, Wilcox defines steady-state conditions to be achieved 
once the discharge temperature remains constant (within sensing accuracy) over a \SI{15}{\minute} interval. 
According to Wilcox, resistance temperature detectors (RTDs)%
\nomenclature[Z]{RTD}{Resistance Temperature Detectors} 
should be used instead of thermocouples wherever possible for improved accuracy. 
Finally, he stresses the importance of recording test data at several different operating points 
to allow recognition of a bad measurement in any one set of data.

\subsection{Compressor Test Data Analysis}
\todo{Summarize Wilcox (and others) compressor performance calculations\dots}

\section{Cycle-Specific Research}
The \IP{1500}{\horsepower} gas test block contains a number of components which must be modeled. 
The compressor is not directly modeled in this work, 
which instead uses external compressor maps or other means 
to specify the discharge conditions given a set of suction conditions. 
The flow measurement orifice, hot gas bypass (HGBP)%
\nomenclature[Z]{HGBP}{Hot Gas Bypass} flow split, condenser, cooling towers, 
throttling expansion devices, and a mixing chamber must be modeled. 
A number of texts and papers were consulted to model these devices.

The flow measurement orifice is modeled according to correlations presented in 
ASME PTC 19.5 \citep{ptc19}. 
This standard presents relationships between 
the pressure drop across the orifice and the flow rate, as shown in \cref{sec:Orifice}. 
Compressible effects are included since the refrigerant will be in a 
superheated vapor state at the orifice.

The flow split, expansion devices, and mixing chamber are modeled using 
basic extensive property balances (mass and energy) as detailed in 
undergraduate thermodynamics texts such as \citet{cengel2011}. 
The equations are developed from the general balance equations 
in \cref{sec:FlowSplit,sec:ExpandMain,sec:ExpandSHV,sec:Mixing}.

\citet{incropera2007} develop equations describing the performance of the condenser, 
which in this case is a shell-and-tube unit. 
The NTU-effectiveness method is used. 
Engineering drawings from JCI were used to determine 
the effectiveness and number of transfer units for the condenser.

\citet{braun1989} developed equations for modeling cooling towers 
using an effectiveness approach analogous to the heat exchanger model.
This is useful as it avoids iteration wherever possible. 
An example useful for verification of the cooling tower model 
is presented by \citet{mitchell2013}. 
A linear regression process is performed on experimental data from 
the BAC cooling towers to determine two characteristic performance parameters, 
which affect the NTU of the cooling tower. 
\citet{braun1989} outline this process and provide several values 
for typical cooling towers. \todo{Need to fit performance data.}

\section{Test Block Design and Construction}
Document York source documents here\dots engineering specifications, 
design changes, pump curves, installed equipment, etc.

\section{Summary}
