\addcontentsline{toc}{chapter}{APPENDICES}
\chapter{Engineering Equation Solver (EES) Code Listing} \label{app:code}

This appendix lists the complete EES code for the thermodynamic model of the test block.
The main EES code is displayed in \cref{lst:main}. This section allows the user to specify
the operating parameters of the desired test and calls additional modules
(\Cref{lst:cmp,lst:orif,lst:split,lst:cond,lst:throttle,lst:clt}) 
which contain 
the bulk of the equations in the model. 
\lstinputlisting[caption={Main EES program code.},label={lst:main}]{code/ees/MainCode.txt}

Alternatively, the user can specify the test parameters and view the model 
predictions in the \emph{Diagram Window} as discussed in \cref{cha:solnmethod}.
One diagram window (\cref{fig:MainDiag}) is for  the complete cycle program,
while the other (\cref{fig:OrifDiag}) is for the orifice selection program.

Many of the modules use several helper functions for common calculations.
These include calculations of entropy and enthalpy for an arbitrary state point
defined by temperature, pressure, and quality. The EES conventions of $x=100$ for
a superheated vapor and $x=-100$ for a subcooled liquid are used in these functions.
A function is also provided for calculation of the Reynolds number.
These functions are shown in \cref{lst:func}.
\lstinputlisting[caption={General helper functions.},label={lst:func}]{code/ees/GenFunctions.txt}

The compressor and orifice modules use additional helper functions.
The compressor module uses helper function \verb+GetX+ to calculate the flow ratio, $X$,
as defined in \cref{eqn:GetX} for the Reynolds number correction.
The orifice uses a function \verb+K_e+ to compute the coefficient for the orifice correlations \cref{eqn:Ke},
as well as a function \verb+GetOrificePressure+ to determine if free or choked flow is applicable \cref{eqn:choked}.
These are displayed in \cref{lst:specfunc}.
\lstinputlisting[caption={Helper functions specific to the compressor and orifice modules.},label={lst:specfunc}]{code/ees/SpecFunctions.txt}

The remainder of the code consists of a \verb+MODULE+ for each device in the cycle.
The assumptions and relationships developed in \cref{cha:modeldev} are 
defined in these modules and applied to general balance equations.
These modules are called by the main code of \cref{lst:main}, and are shown in \crefrange{lst:cmp}{lst:clt}.

The compressor module is displayed in \cref{lst:cmp}.
\lstinputlisting[caption={Compressor module of EES implementation.},label={lst:cmp}]{code/ees/Cmp.txt}

The orifice module is displayed in \cref{lst:orif}.
\lstinputlisting[caption={Orifice module of EES implementation.},label={lst:orif}]{code/ees/orif.txt}

The flow split module is displayed in \cref{lst:split}.
\lstinputlisting[caption={Flow split module of EES implementation.},label={lst:split}]{code/ees/split.txt}

The condenser module is displayed in \cref{lst:cond}.
\lstinputlisting[caption={Condenser module of EES implementation.},label={lst:cond}]{code/ees/cond.txt}

The expansion device module is displayed in \cref{lst:throttle}.
This module is called for both the condensed stream and bypassed stream.
\lstinputlisting[caption={Expansion device module of EES implementation.},label={lst:throttle}]{code/ees/throttle.txt}

The cooling tower module is displayed in \cref{lst:clt}.
\lstinputlisting[caption={Cooling tower module of EES implementation.},label={lst:clt}]{code/ees/clt.txt}