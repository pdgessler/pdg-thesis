\chapter{Results and Discussion} \label{cha:results}
A crucial component of any modeling effort is to confirm
that the model agrees with experimental data to a level
appropriate for the intended application.
This validation process confirms the appropriateness of the simplifying
assumptions made during the model development.
For the \IP{1500}{\horsepower} gas test block, experimental data are available for 
numerous tests conducted in July 2005 and November 2012.
The test data from JCI were obtained using the guidelines
in \textcite{bluebook} and in accordance with \textcite{ptc10},
as described in \cref{sec:TestMethodology}.
A subset of the test data used for validation purposes
is shown in \cref{tab:SampleData} of \cref{app:sample}.

The complete cycle model
was run with inputs matching each experimental data set
to validate the model over a wide range of operating conditions.
Because outdoor air conditions were not recorded at the time of the test,
the conditions have been sourced from weather data archives \parencite{wunder2005,wunder2012}.
The resulting model output variables are compared with the
experimental data in \cref{sec:expresults}.
Parity plots are used to illustrate the agreement 
between the model results and experimental data.
The results agree to within 2.5\% in the compressor and orifice modules,
and to within 2.0\% in the condenser and cooling tower modules,
which required broader assumptions.

\Cref{sec:sensitivity} studies the effect of intermediate
variables on the overall results.
For example, subtle differences between the calculation methods used
in the model and used during experiments mean that the Reynolds numbers
for the compressor and the orifice
do not agree with the experimental data nearly as well
as other variables.
However, while the Reynolds numbers have a subtle effect on the 
compressor discharge state and the orifice
differential pressure, the sensitivity analysis
reveals that the overall results are not significantly affected.
Therefore, the sensitivity analysis is an essential step in
proving the robustness of the model.

\section{Comparison of Experimental and Model Results} \label{sec:expresults}
This section presents the validation results for the complete cycle program.
A sample set of detailed results is shown in \cref{tab:SampleData}.
Based on the data collected by JCI during each test, the compressor,
orifice, and condenser module results from the model may be directly compared
to experimental data.
The results for each condition considered are summarized in 
parity plots, \crefrange{fig:DischargePressureParity}{fig:AvgCondWatTempParity},
alongside related discussion.
Each parity plot illustrates perfect agreement between  modeled 
and experimental results with a solid line and $\pm 5\%$
difference bounds with dashed lines, unless otherwise noted.

\subsection{Compressor Module} \label{sec:CompResults}
Because the compressor module is based on compressor
characteristics that are determined through experimental testing, 
the results are expected to show excellent agreement.
Indeed, the maximum percent difference for the compressor discharge pressure is 0.018\%,
as shown in \cref{fig:DischargePressureParity}.
Similarly, the maximum difference for the compressor discharge temperature 
(\cref{fig:DischargeTemperatureParity}) is \IP{1.0}{\fahrenheit}, or 0.2\% by using an absolute temperature scale.
The error is still acceptable in this case, but is larger because of a difference
in the compressor Reynolds number computation.
\begin{figure}[tbp]
  \centering
  \includegraphics{figs/DischargePressureParity}
  \caption{Compressor discharge pressure parity plot.}
  \label{fig:DischargePressureParity}
\end{figure}
\begin{figure}[tbp]
  \centering
  \includegraphics{figs/DischargeTemperatureParity}
  \caption{Compressor discharge temperature parity plot.}
  \label{fig:DischargeTemperatureParity}
\end{figure}

The undetermined difference in calculation methods results in an average of 12\%
difference between the compressor Reynolds number calculated in the model and calculated during experiments,
as shown in \cref{fig:CompReParity}.
This impacts the Reynolds number correction (\cref{eqn:ReCorr}) used
in the calculation of isentropic efficiency.
Because the model predicts a larger Reynolds number than was calculated during
experiments, the model also will predict a greater isentropic efficiency.
Therefore, the model will tend to slightly underpredict the
discharge temperature, as seen in \cref{fig:DischargeTemperatureParity}.
The sensitivity of the model to this Reynolds number and its impact on
model outputs is studied further in \cref{sec:sensitivity}.
\begin{figure}[tbp]
  \centering
  \includegraphics{figs/CompressorReParity}
  \caption{Compressor Reynolds number parity plot.}
  \label{fig:CompReParity}
\end{figure}

The compressor gas horsepower is shown in \cref{fig:PowerParity}.
This required output of the model agrees with the experimental results
to within 2.5\%.
The compressor Reynolds number calculation affects the gas horsepower slightly,
because the lower discharge temperature predicted by the model
corresponds to a lower discharge enthalpy,
and the gas horsepower is directly proportional to the increase in enthalpy
across the compressor (\cref{eqn:CompEnergy}).
Another possible source of error is the assumption of an adiabatic
compressor.
Heat transfer from the compressor casing to the ambient environment
would increase the gas horsepower requirement for the test block.
\begin{figure}[tbp]
  \centering
  \includegraphics{figs/CompressorPowerParity}
  \caption{Compressor gas horsepower parity plot.}
  \label{fig:PowerParity}
\end{figure}

\subsection{Orifice Module} \label{sec:OrifResults}
While the temperature at the orifice has only a slight effect
on the differential pressure correlations (\cref{sec:Orifice}),
there is a significant length of refrigerant piping
between the orifice flow meter and the compressor discharge port.
Therefore, comparing the model's predicted orifice temperature to the 
measured data provides a means to evaluate the assumption
of negligible heat transfer from piping on the test block.
The results for orifice temperature have a maximum
difference of \IP{2}{\fahrenheit} or 0.5\% on an absolute temperature scale, 
as shown in \cref{fig:OrificeTempParity}.
\begin{figure}[tbp]
  \centering
  \includegraphics{figs/OrificeTempParity}
  \caption{Orifice temperature parity plot.}
  \label{fig:OrificeTempParity}
\end{figure}
Because the model consistently predicts a higher temperature
than was measured, it is expected that heat transfer to the
ambient environment is the cause of this difference.
However, the orifice differential pressure (\cref{fig:OrificeDiffParity})
is the important output in this area of the model.
The impact of this temperature difference on the differential pressure is negligible,
so the assumption is deemed appropriate.

The orifice differential pressure is the primary output of the model
in the orifice selection program.
The differential pressure is used to evaluate the suitability
of different orifice diameters for the range of test conditions specified.
As shown in \cref{fig:OrificeDiffParity}, the model shows excellent 
agreement with the experimental results, with a maximum percent difference
of 0.35\% for the conditions considered for validation.
This shows that the orifice selection program will be an excellent 
aid to simplify the calculations required to plan a set of tests.
\begin{figure}[tbp]
  \centering
  \includegraphics{figs/OrificeDiffPressureParity}
  \caption{Orifice differential pressure parity plot.}
  \label{fig:OrificeDiffParity}
\end{figure}

\subsection{Condenser Module} \label{sec:condresult}
In the condenser, the outlet pressure is recorded in the experimental data.
This allows the constant pressure assumptions in the flow split (\cref{sec:FlowSplit})
and condenser (\cref{sec:Condenser}) modules to be evaluated.
As shown in \cref{fig:CondPressParity}, the model results for condenser pressure
deviate from the experimental results by a maximum of 0.15\%.
Therefore, the constant pressure assumptions in the components
and connecting piping are reasonable for the \IP{1500}{\horsepower} gas test block.
\begin{figure}[tbp]
  \centering
  \includegraphics{figs/CondenserPressureParity}
  \caption{Condenser pressure parity plot.}
  \label{fig:CondPressParity}
\end{figure}

The condenser refrigerant liquid temperature is also recorded,
and it can be used to calculate the degree of subcooling in the condenser.
The comparison of modeled and measured liquid temperature is 
used to confirm that the condenser model provides an
accurate representation of the complex heat transfer taking place.
\Cref{fig:CondRefLiqTempParity} shows the results for the condenser
refrigerant liquid temperature, which have a maximum deviation from
experimental results of \IP{1.5}{\fahrenheit} or 0.5\% on an absolute temperature scale.
\begin{figure}[tbp]
  \centering
  \includegraphics{figs/CondenserRefrigerantLiquidTempParity}
  \caption{Condenser refrigerant liquid temperature parity plot.}
  \label{fig:CondRefLiqTempParity}
\end{figure}

\subsection{Cooling Tower Module}
The average condenser water temperature, which is also the 
average water temperature in the cooling tower, provides insight
into the performance of both the cooling tower and condenser.
\Cref{fig:AvgCondWatTempParity} shows a comparison of model and
experimental results.
Despite many assumptions concerning the parameters used in the condenser
and cooling tower models, such as $UA$, $c$, and $n$ (\cref{sec:Condenser,sec:CoolingTower}),
the results agree with the experimental data to within \IP{5}{\fahrenheit}
or 2\% on an absolute temperature scale.
\begin{figure}[tbp]
  \centering
  \includegraphics{figs/AvgCondenserWaterTempParity}
  \caption{Average condenser water temperature parity plot.}
  \label{fig:AvgCondWatTempParity}
  \vspace{-\baselineskip}
\end{figure}

Because the parameters are currently empirically-based,
these differences are liable to increase at operating conditions
further away from the conditions recorded in the experimental data.
Recall from \cref{sec:CoolingTower} that several cooling towers on
the physical test block are lumped into a single cooling tower for
the purposes of the model.
In addition, neither the cooling tower fan speeds nor the condenser
water flow rate are recorded as part of the test data, so average
values are currently used.
This is discussed further in \cref{sec:sensitivity}, and
several of the recommendations discussed in \cref{cha:conclusion}
directly relate to additional data collection in these areas of the test
facility.
These improvements would lead to enhanced parameter selection
in this area of the model.

\section{Sensitivity Analysis} \label{sec:sensitivity}
This section shows that the differences between
Reynolds numbers calculated in the compressor and orifice modules 
(\cref{sec:CompResults,sec:OrifResults}) and calculated based on
experiments have a negligible
impact on the important output variables.
It also studies the impact of the condenser water flow rate on the
water temperatures and cooling tower flow rate requirements,
because of the uncertainty involved in estimating the water flow rate.
The model currently uses an average water flow rate calculated
from test data.
This is because a bypass loop is used on the block, but no data
on the bypass loop flow rate or condenser loop flow rate are currently recorded.

In the compressor, the Reynolds number is used in the calculation
of the isentropic efficiency, which affects the discharge temperature
and thus the gas horsepower of the compressor.
The discharge pressure should not be affected.
To study the extent of these effects, the Reynolds number
was artificially adjusted by several different percentages,
and the compressor module outputs were recorded.
These results are listed in \cref{tab:SensComp}.
\begin{table}[tbp]
  \centering
  \caption{Sensitivity analysis results for the compressor Reynolds number, $\RE$.
    The reference Reynolds number for this condition is $\RE=\num{2.591e6}$.}
  \label{tab:SensComp}%
  \begin{tabular}{SSSS}
    \toprule
    {Adjustment factor} & {Discharge pressure,}   & {Discharge temperature,}   & {Gas horsepower,} \\ 
    {for calculated $\RE$} & {$p_d$ (\ip{\psia})} & {$T_d$ (\ip{\fahrenheit})} & {$\dot{W}_{\text{gas}}$ (\ip{\horsepower})} \\
    \midrule
    0.85  & 99.87 & 129.2 & 438.7 \\
    0.90  & 99.87 & 129.1 & 438.2 \\
    0.95  & 99.87 & 129.0 & 437.8 \\
    1.00  & 99.87 & 129.0 & 437.4 \\
    1.05  & 99.87 & 129.0 & 437.1 \\
    1.10  & 99.87 & 129.0 & 436.7 \\
    1.15  & 99.87 & 128.9 & 436.4 \\
    \bottomrule
  \end{tabular}%
\end{table}%
As expected, the discharge pressure is not affected.
The discharge temperature is affected by a maximum of \IP{0.2}{\fahrenheit},
while the gas horsepower is affected by a maximum of \IP{1.3}{\horsepower}.
These results show that the effect of the Reynolds number discrepancy on the outputs is very slight.
Therefore, the error in the compressor gas horsepower (\cref{fig:PowerParity})
is mostly attributable to the adiabatic assumption of \cref{sec:Compressor}.

The Reynolds number calculated at the orifice shows the same discrepancy as 
the calculation at the compressor.
In the orifice, the Reynolds number has a slight effect on the coefficient
of discharge, $C$, calculated using the ASME PTC~19.5 correlations \parencite{ptc19} of
\cref{sec:Orifice}.
The orifice Reynolds number was adjusted to study the significance of these effects.
The effects on the orifice module are even less significant than the effects 
on the compressor module.
\Cref{tab:SensOrif} displays the results, and it is clear that the effect
is well under the level of uncertainty in the measurements on the gas test block.
\begin{table}[tbp]
  \centering
  \caption{Sensitivity analysis results for the orifice Reynolds number, $\RE$.
    The reference Reynolds number for this condition is $\RE=\num{7.891e6}$.}
  \label{tab:SensOrif}%
  \begin{tabular}{SSS}
    \toprule
    {Adjustment factor}    & {Outlet temperature,}           & {Orifice differential} \\
    {for calculated $\RE$} & {$T_{\out}$ (\ip{\fahrenheit})} & {pressure, $\Delta p_{\text{orif}}$ (\ip{\psid})} \\
    \midrule
    0.85  & 128.37291 & 3.11289 \\
    0.90  & 128.37290 & 3.11294 \\
    0.95  & 128.37289 & 3.11298 \\
    1.00  & 128.37288 & 3.11303 \\
    1.05  & 128.37288 & 3.11306 \\
    1.10  & 128.37287 & 3.11310 \\
    1.15  & 128.37286 & 3.11313 \\
    \bottomrule
  \end{tabular}
\end{table}%
Therefore, the discrepancy in Reynolds number calculation is clearly negligible for the orifice module.

In the condenser, an average water flow rate calculated using experimental results
is used in the absence of true flow rate measurements.
The distribution of water flow between the condenser loop and the bypass loop is unknown.
Changes in the water flow rate will affect the water temperatures at the condenser and,
most importantly, the required air mass flow rate in the cooling tower.
\Cref{tab:SensCond} shows the effects of changing the condenser water flow rate
on the condenser water temperature difference, average water temperature, and also the 
cooling tower air mass flow rate.
\begin{table}[tbp]
  \centering
  \caption{Sensitivity analysis results for the condenser water mass flow rate, $\dot{m}_{\wat}$.
    The reference water flow rate for this condition is $\dot{m}_{\wat}=\IP{6768}{\poundmass\per\minute}$.}
  \label{tab:SensCond}%
  \begin{tabular}{SSSS}
    \toprule
    {Adjustment factor}              & {Water temperature}                                & {Average water}                                    & {Cooling tower air flow} \\
    {for flow rate $\dot{m}_{\wat}$} & {difference, $\Delta T_{\wat}$ (\ip{\fahrenheit})} & {temperature, $T_{\text{avg}}$ (\ip{\fahrenheit})} & {rate, $\dot{m}_{\air}$ (\ip{\poundmass\per\minute})} \\
    \midrule
    0.85  & 3.227 & 70.38 & 917.4 \\
    0.90  & 3.048 & 70.29 & 923.0 \\
    0.95  & 2.888 & 70.21 & 928.3 \\
    1.00  & 2.743 & 70.14 & 933.1 \\
    1.05  & 2.613 & 70.07 & 937.6 \\
    1.10  & 2.494 & 70.02 & 941.7 \\
    1.15  & 2.385 & 69.96 & 945.6 \\
    \bottomrule
  \end{tabular}%
\end{table}%
For a 30\% change in water flow rate, the cooling tower air flow rate changes
by only 3\%.
The water temperatures are affected slightly, but overall,
the cooling tower mass flow rate is fairly insensitive to the 
specified condenser water mass flow rate.
