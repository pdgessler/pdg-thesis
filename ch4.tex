\chapter{Solution Methodology} \label{cha:solnmethod}

The equations developed in \cref{cha:modeldev} are implemented in
Engineering Equation Solver (EES). 
This software package provides thermophysical property data, unit checking, 
a graphical user interface (GUI) framework, and an automatic equation 
blocking scheme for iteratively solving systems of simultaneous equations. 
These features, along with compressor and test engineers' familiarity with the program, 
make EES an ideal choice for implementation of the gas test block model.

The model is implemented using an EES module for each device in the cycle. 
As much as possible, the equations and assumptions for each device are contained
within the corresponding module.
This facilitates a semantic calling structure in the main body of the program 
and will simplify code maintenance for future modifications or improvements 
to the model.
The overall structure of the implementation is described in \cref{sec:Overall}.

One of the goals of the project, the ability to select the best flow measurement
orifice for a user-input range of operating points, requires multiple operating points
to be evaluated.
The remainder of the goals consist of calculations for a single operating point at a time.
A limitation of the EES software restricts the number of variables stored.
Therefore, it is not feasible to store the results for the entire cycle at multiple operating points.
For this reason, two EES programs were developed, each sharing a common code base. 

The first program computes results for the entire cycle, but only 
for a single operating point at a time.
This program is used for all use cases except orifice selection
and is discussed in \cref{sec:MainModel}.
The second program has the capability to evaluate and store orifice and 
compressor model results for multiple operating points.
It calls only the compressor and orifice modules of the code to minimize the number of variables required.
This model is used for orifice selection and is discussed in \cref{sec:OrifSelect}.

\section{Implementation Structure} \label{sec:Overall}
As discussed in the opening of \cref{cha:solnmethod}, the code is organized
in a way that mimics the physical connection of the devices in the cycle.
Wherever possible, the balance equations, related assumptions, and any required
correlations are placed into an EES module.
The inputs of these modules represent the thermodynamic state of physical flow streams entering
the device, and the outputs represent that of the flow streams exiting the component.
This makes the calling structure intuitive and allows for improvements in the
modeling assumptions for a given component without disrupting the rest of the model.

The complete EES code is displayed and briefly explained in \cref{app:code}.
The modules do not refer to state numbers (from, for example, \cref{fig:SimpSchematic})
so that additional components can be added in the future with minimal disruption.
Only the indices of the top-level array variables correspond to the state numbering.
This code is used as the basis for both the complete cycle program and the 
orifice selection program. 
Each of these programs use EES' diagram window functionality
to provide a user interface for the model.

\section{Complete Cycle Program} \label{sec:MainModel}
The diagram window for the complete cycle program is shown in \cref{fig:MainDiag}.
The required inputs are grouped into three clusters: the outdoor air conditions
at the upper left, the block configuration settings (including orifice index
as listed in \cref{tab:OrifData}) at the upper right, and
the compressor operating point parameters to the left of the compressor.
The compressor suction conditions are specified just below the compressor
on the diagram.
\begin{figure}[htbp]
  \centering
  \includegraphics[width=\textwidth]{figs/MainDiagram}
  \caption{Engineering Equation Solver (EES) Diagram Window for the complete
    cycle program. User inputs are boxed numbers, while program outputs are displayed without boxes.}
  \label{fig:MainDiag}
\end{figure}

After specifying the inputs, the program may be run using the `Calculate' button. 
Updated results will then appear.
The output states and other calculated values are displayed near the corresponding
state point on the diagram. 
The results may also be viewed using any of EES' other functionality.
In addition, the model may be run directly from the Equations Window or
from a Parametric Table to study multiple operating conditions.

\section{Orifice Selection Program} \label{sec:OrifSelect}
The orifice selection program calls only the 
compressor and orifice modules of the main code.
These modules are all that is required to model the orifice differential pressure,
which is used to evaluate different flow measurement orifices over a specified
range of test conditions.
This simplification keeps the orifice selection program within the maximum
number of variables allowed by EES.
A flowchart of the orifice selection algorithm is shown in \cref{fig:FlowOrifSelect}.
\begin{figure}[htbp]
  \centering
  \includegraphics{figs/OrificeSelectionFlowchart}
  \caption{Flowchart for orifice selection procedure. 
    The boxed section is executed for each flow coefficient, $\Theta$,
    in a range specified by the user.}
  \label{fig:FlowOrifSelect}
\end{figure}

The compressor module is called once to establish the discharge conditions
based on suction conditions and compressor map parameters.
Then, the orifice differential pressure is evaluated and stored for each orifice
diameter and its corresponding pipe diameter.
These diameters are stored in an EES lookup table, which allows for future modifications.
The orifice numbering scheme and orifice data are shown in \cref{tab:OrifData}.
This calculation process is repeated for each flow coefficient
in a range $\Theta_{\min} \leq \Theta \leq \Theta_{\max}$.
\begin{table}[htbp]
  \centering
  \caption{Orifice numbering and diameter data \parencite{bluebook}. 
    This table corresponds to the information stored in the EES lookup table `\modu{OrificeData}'.}
  \label{tab:OrifData}
  \begin{tabular}{lSS}
  \toprule
  Orifice & {$d$ (\si{\inch})} & {$D_{\text{pipe}}$ (\si{\inch})} \\
  \midrule
  1 & 6.250 & 7.981 \\
  2 & 5.000 & 7.981 \\
  3 & 3.504 & 7.981 \\
  \midrule
  4 & 4.000 & 6.065 \\
  5 & 3.750 & 6.065 \\
  6 & 3.080 & 6.065 \\
  7 & 3.002 & 6.065 \\
  8 & 2.500 & 6.065 \\
  9 & 1.999 & 6.065 \\
  \midrule
  10& 3.750 & 5.047 \\
  11& 3.123 & 5.047 \\
  12& 2.473 & 5.047 \\
  13& 1.999 & 5.047 \\
  14& 1.250 & 5.047 \\
  15& 1.000 & 5.047 \\
  \bottomrule
  \end{tabular}
\end{table}

The diagram window for the orifice selection program is shown in \cref{fig:OrifDiag}.
The number of flows evaluated within the range is set in the equations window so
that the program does not exceed EES' maximum number of variables.
If, for some reason, this limit is exceeded, proper operation is restored
by selecting `Options'$\to$`Purge Unused Variables' and selecting `Yes' in the dialog box that appears.
These samples are evenly distributed throughout the user-specified range of flow coefficients.
\begin{figure}[htbp]
  \centering
  \includegraphics[width=0.75\textwidth]{figs/OrifDiagram}
  \caption{Engineering Equation Solver (EES) Diagram Window for the orifice
    selection program. User inputs are boxed numbers, while program outputs 
    are displayed without boxes.}
  \label{fig:OrifDiag}
\end{figure}

\todo{floats are piling up here. Address placement after the text settles down.}
After calculating the results, an EES plot is used to visualize
the differential pressure and evaluate orifices for the specified flow conditions.
The results are plotted on semi-logarithmic axes for clarity at both high and low
differential pressures.
A typical set of results is shown in \cref{fig:OrifResults}.
The ordinate axis limits correspond to the acceptable differential pressure range
for flow measurement, \SIrange{10}{990}{\inch\ce{H2O}}, or \SIrange{0.36}{35.70}{\psid}.
So a particular orifice is acceptable if its differential pressure curve remains 
within the axis limits over the desired range of flow coefficients.
\begin{figure}[htbp]
  \centering
  \includegraphics[width=\textwidth]{figs/OrifResults}
  \caption{Typical orifice selection program results. The axis limits correspond
    to the \SIrange{10}{990}{\inch\protect\ce{H2O}} acceptable measurement range.}
  \label{fig:OrifResults}
\end{figure}

As discussed in \cref{sec:Orifice}, the model predicts choked flow to produce a
converged result for all orifices at all flow coefficients.
However, on the gas test block, choked flow would limit the mass flow rate, so
the specified operating point cannot be achieved with a particular orifice
if choked flow is predicted.
Some results for choked flow are shown in \cref{fig:OrifChoked}.
For this scenario, orifices 1 and 2 do not reach choked conditions, while
orifices 3--15 do reach choked flow.
In this case, only orifice 2 is acceptable, since orifice 1 is outside
the accepted bounds at the low end of the range of flow coefficients.
\begin{figure}[htbp]
  \centering
  \includegraphics[width=\textwidth]{figs/OrifChoked}
  \caption{Orifice selection program results showing choked flow for orifices 3--15. 
    For these conditions, only orifice 2 is acceptable over the entire range of flow coefficients.}
  \label{fig:OrifChoked}
\end{figure}

\emph{PDG: How to end this chapter? It seems awfully abrupt as-is. Am I missing something?}