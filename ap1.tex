\addcontentsline{toc}{chapter}{APPENDICES}
\chapter{Engineering Equation Solver (EES) Code Listing} \label{app:code}

This appendix lists the complete EES code for the numerical model of the test block.
The main EES file is listed in \cref{lst:main}. This file allows the user to specify
the operating parameters of the desired test and \verb+$INCLUDE+s additional files
%(\Cref{lst:assume,lst:const,lst:ebal,lst:mbal,lst:orifice,lst:prop}) 
which contain 
the bulk of the equations in the model. \note{This must be updated manually (and is currently out-of-date). 
(\texttt{.ees} files are not plaintext.)}
%{\lstinputlisting[caption={Main EES Program},label={lst:main}]{code/ees/CompressorBlock_Diagram.txt}}
Alternatively, the user can specify the test parameters and view the model 
predictions in the \emph{Diagram Window}. A screenshot of the EES Diagram Window 
for the test block model is shown in \cref{fig:DiagWindow}. The test parameters
set in the Diagram Window are used if the diagram window is open, otherwise the
settings in the main EES program 
%(\Cref{lst:main}) 
are used.
\begin{figure}[htbp]
	\centering
	PLACEHOLDER FOR DIAGRAM WINDOW SCREENSHOT
	\caption{EES Diagram Window user interface for test block model.}
	\label{fig:DiagWindow}
\end{figure}

Although EES has many fundamental constants built-in, a few additional constants
were required. These are collected in \cref{lst:const} and are simply for convenience
in writing the remainder of the equations.
%\lstinputlisting[float=htbp,caption={Physical Constants},label={lst:const}]{code/ees/Constants.txt}

The engineering assumptions used to reduce the fundamental governing differential
equations to a tractable form are listed in \cref{lst:assume}. \Cref{cha:modeldev}
provides a detailed explanation of each assumption and its implications.
%\lstinputlisting[float=htbp,caption={Modeling Assumptions},label={lst:assume}]{code/ees/Assumptions.txt}

The thermodynamic states in the model are `fixed' by knowledge of two independent
properties. Based on these two property values, the rest of the properties at the
state are calculated using built-in EES functions. These property calculations are
presented in \cref{lst:prop}. \todo{Replace all this code with some state lookup 
function. (Lots of repeated statements at the moment.)}
%\lstinputlisting[caption={Fluid Property Lookups},label={lst:prop}]{code/ees/PropertyLookups.txt}

The orifice flow meter is modeled as described in \cref{sec:Orifice}. The EES
implementation of the equations is shown in \cref{lst:orifice}.
%\lstinputlisting[float=htbp,caption={Orifice Flow Meter Model},label={lst:orifice}]{code/ees/OrificeModel.txt}
The mass balances and energy balances for each device in the test block cycle
are listed in \cref{lst:mbal,lst:ebal}, respectively.
%\lstinputlisting[float=htbp,caption={Mass Balances},label={lst:mbal}]{code/ees/MassBalances.txt}
%\lstinputlisting[float=htbp,caption={Energy Balances},label={lst:ebal}]{code/ees/EnergyBalances.txt}