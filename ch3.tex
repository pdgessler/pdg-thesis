\chapter{Theoretical Model Development} \label{cha:modeldev}
This chapter presents the development of the equations used to model 
the compressor test block from first principles. 
\Cref{sec:GenBal} contains the general mass and energy balance equations, 
to which the appropriate assumptions for each device in the cycle are applied 
to produce the device-specific equations, as detailed in \crefrange{sec:Compressor}{sec:Mixing}. 
Subscripts $\inl\nomenclature[X]{$\inl$}{Inlet conditions}$ 
and $\out\nomenclature[X]{$\out$}{Outlet conditions}$ 
are used to indicate the inlet and outlet states of the device, respectively. 

\section{General Balance Equations} \label{sec:GenBal}
The rate forms of the balance equations are used throughout the model.
This section lists the mass and energy balances in their most general form \parencite{cengel2011}, 
so that engineering assumptions may be applied on a per-device basis.

The general mass balance is obtained by 
applying conservation of mass to the control volume. 
This process yields
\begin{equation}
	\dod{m_\CV}{t} = \sum_{\text{inlets}}{\dot{m}_\inl} - \sum_{\text{outlets}}{\dot{m}_\out},
  \label{eqn:GenMass}
	\nomenclature{$t$}{Time\nomunit{\minute}} 
  \nomenclature{$m$}{Mass\nomunit{\poundmass}} 
	\nomenclature[X]{$\CV$}{Control volume}
\end{equation}
where the summations account for any number of inlets and outlets to the device.
\Cref{eqn:GenMass} includes only transport terms since mass production is 
always identically zero (neglecting nuclear processes).
This result is applied to each device in the cycle, with the appropriate number of inputs and outputs.

Similarly, the general energy balance is obtained by writing 
an extensive property balance over the control volume. 
By the first law of thermodynamics, energy production is identically zero 
(again, neglecting nuclear processes), so only transport terms remain in the equation. 
However, heat transfer and work transfer interactions, and kinetic and potential energy changes
can change the energy of the control volume in addition to the transport terms, so
\begin{equation}
	\dod{E_\CV}{t} = \dot{Q} - \dot{W} + \sum_{\text{inlets}}{(\dot{m}h^{\circ})_\inl} - 
		\sum_{\text{outlets}}{(\dot{m}h^{\circ})_\out},
  \label{eqn:GenEnergy}
	\nomenclature{$e$}{Specific energy, $e=u+\text{ke}+\text{pe}$\nomunit{\btu\per\poundmass}} 
	\nomenclature{$Q$}{Heat transfer\nomunit{\btu}} 
	\nomenclature{$W$}{Work transfer\nomunit{\btu}} 
	\nomenclature[Y]{$\dot{\left(\;\right)}$}{Quantity per unit time or rate, $\dif{\left(\;\right)}/\mkern-3mu\dif{t}$\nomunit{$\left[\;\right]$\per\minute}} 
	\nomenclature{$p$}{Pressure\nomunit{\psia,\,\psig,\,\psid}}
	\nomenclature{$v$}{Specific volume\nomunit{\foot\cubed\per\poundmass}}
	\nomenclature{$u$}{Specific internal energy\nomunit{\btu\per\poundmass}}
  \nomenclature[A$z$]{*}{Unless otherwise noted, symbols for extensive properties are the uppercase variants of symbols used for the corresponding specific (intensive) properties.}
\end{equation}
where the \emph{methalpy}, $h^\circ = h+\mathrm{ke}+\mathrm{pe}%
  \nomenclature{$h^{\circ}$}{Specific methalpy, $h^\circ=h+\mathrm{ke}+\mathrm{pe}$\nomunit{\btu\per\poundmass}}$, 
is a convenient way to account for 
the enthalpy, $h\nomenclature{$h$}{Specific enthalpy, $h=u+pv$\nomunit{\btu\per\poundmass}}$, and
kinetic and potential energy effects if required. 
The kinetic energy can be calculated as
\begin{equation}
  \mathrm{ke} = \vel^2/2,
  \nomenclature[A$k$]{$\mathrm{ke}$}{Specific kinetic energy, $\mathrm{ke}=\vel^2/2$\nomunit{\btu\per\poundmass}}
  \nomenclature[A$V$]{$\vel$}{Velocity magnitude\nomunit{\foot\per\minute}}
\end{equation}
where the average velocity $\vel$ can be calculated using 
$\dot{m}=\rho \vel A%
  \nomenclature[W]{$\rho$}{Density, $1/v$\nomunit{\poundmass\per\foot\cubed}}%
  \nomenclature{$A$}{Area\nomunit{\foot\squared,\,\inch\squared}}$. The potential energy
can be calculated as
\begin{equation}
  \mathrm{pe} = gz,
	\nomenclature[A$p$]{$\mathrm{pe}$}{Specific potential energy, $\mathrm{pe}=gz$\nomunit{\btu\per\poundmass}}
	\nomenclature{$g$}{Gravitational acceleration, $g=\IP{115826}{\foot\per\minute\squared}$\nomunit{\foot\per\minute\squared}}
	\nomenclature{$z$}{Elevation\nomunit{\foot}}
\end{equation}
where $g$ is the gravitational acceleration 
and $z$ is the elevation of the inlet/outlet.

\Cref{eqn:GenEnergy} shows the standard sign convention used throughout this thesis; 
that is, heat transfer \emph{into} the system is considered positive, 
while work transfer \emph{out of} the system is considered positive.
\Cref{eqn:GenMass,eqn:GenEnergy} form the basic equations to which 
engineering assumptions are applied in \crefrange{sec:Compressor}{sec:Mixing}.
Assumptions common to each device in the cycle are that
\begin{enumerate}
  \item the steady-flow condition eliminates both time derivatives $\dif{m_\CV}/\!\dif{t}$ and $\dif{E_\CV}/\!\dif{t}$, \label{lst:sf}
  \item changes in potential energy are neglected, so $\Delta \text{pe} = 0$, and \label{lst:pe}
  \item changes in kinetic energy are neglected, so $\Delta \text{ke} = 0$. \label{lst:ke}
\end{enumerate}
As a consequence of assumptions~\ref{lst:pe} and~\ref{lst:ke},
the methalpy notation reduces to the more familiar enthalpy form,
and assumption~\ref{lst:sf} means that the left-hand sides of \cref{eqn:GenMass,eqn:GenEnergy} both become zero:
\begin{align}
  0 &= \sum_{\text{inlets}}{\dot{m}_{\inl}} - \sum_{\text{outlets}}{\dot{m}_{\out}}, \label{eqn:SemiMass}\\
  0 &= \dot{Q} - \dot{W} + \sum_{\text{inlets}}{(\dot{m}h)_{\inl}} - \sum_{\text{outlets}}{(\dot{m}h)_{\out}}. \label{eqn:SemiEnergy}
\end{align}
\cref{eqn:SemiMass,eqn:SemiEnergy} are used as the basis for all component models
in \crefrange{sec:Compressor}{sec:Mixing}.

\section{Compressor} \label{sec:Compressor}
Detailed modeling of the compressor is complex and worthy of a dissertation in its own right. 
To make matters worse, the model should make reasonable predictions 
of a compressor test for \emph{any} compressor, 
whether already prototyped or in the early stages of development.
Fortunately, detailed compressor models 
for predicting compressor performance are not required for the purposes of test block modeling. 
Instead, the performance of the compressor is calculated by the model using
data from compressor maps. 
An example of a typical compressor map is shown in \cref{fig:GenMap}. 
\begin{figure}[tbp]
  \centering
  \includegraphics{figs/GenericCompMap}
  \caption{Typical compressor map, showing flow and head axes, speed lines, and efficiency islands.
    The operating region for the compressor is the region bounded by the surge and choke lines.}
  \label{fig:GenMap}
\end{figure}
These compressor maps are developed by JCI either 
experimentally for existing compressors or numerically%
---using other compressor design tools, from basic one-dimensional methodologies
to full three-dimensional CFD---for new compressor designs.

A user of the model can select an operating point on the compressor map, 
and with specification of the suction (inlet) conditions, 
the discharge (outlet) state of the compressor is fixed. 
The compressor maps use the dimensionless coefficients of flow, head, velocity, and efficiency
($\Theta$, $\Omega$, $\MA$, and $\eta_s$, respectively)  
to generalize machine characteristics. 
The flow coefficient, $\Theta$, \parencite{trevino2012} is given by
\begin{equation}
  \Theta = \frac{\dot{V}}{\svel D^2},
	\nomenclature{$\svel$}{Acoustic (sonic) velocity\nomunit{\foot\per\minute}}
	\nomenclature[W]{$\Theta$}{Flow coefficient, $\dot{V}/{\svel D^2}$\nomunit{-}}
\end{equation}
where $\dot{V}$ is the volumetric flow rate, 
$\svel$ is the acoustic (sonic) velocity at total suction conditions, 
and $D$ is the impeller tip diameter.
The head coefficient, $\Omega$, \parencite{trevino2012} is given by
\begin{equation}
	\Omega = \frac{g_c \Delta h_s}{\svel^2},
	\nomenclature[W]{$\Omega$}{Isentropic head coefficient, ${g_c \Delta h_s}/{\svel^2}$\nomunit{-}}
	\nomenclature{$g_c$}{Gravitational proportionality constant, $g_c=\IP{115826}{\poundmass\foot\per\poundforce\minute\squared}$\nomunit{\poundmass\foot\per\poundforce\minute\squared}}
\end{equation}
where $g_c$ is the gravitational proportionality constant, 
$\Delta h_s$ is the specific enthalpy change for an isentropic compression process, 
and $\svel$ is, again, the acoustic velocity at total suction conditions.

The Machine Mach number, $\MA$, is 
\begin{equation}
  \MA = \frac{\vel_{\text{tip}}}{a},
  \nomenclature[A$Ma$]{$\MA$}{Machine Mach number, $\vel_{\text{tip}}/a$\nomunit{-}}
\end{equation}
where $\vel_{\text{tip}}$ is the impeller tip velocity
and $a$ is the acoustic velocity at total inlet (suction) conditions \parencite{trevino2012}.
The third and final compressor performance characteristic gleaned 
from the compressor map is the isentropic efficiency, $\eta_s$, 
which quantifies the deviation of the real compression process 
from an isentropic (adiabatic and reversible) compression process. 
The isentropic efficiency \parencite{cengel2011} is defined as
\begin{equation}
  \eta_s = \frac{h_{\out s}-h_{\inl}}{h_{\out}-h_{\inl}},
  \nomenclature[X]{$s$}{Isentropic} 
  \nomenclature[W]{$\eta$}{Efficiency\nomunit{-}}
\end{equation}
where $h_{\out s}$ is the discharge enthalpy for an isentropic compression process, 
$h_{\inl}$ is the suction enthalpy, 
and $h_{\out}$ is the actual discharge enthalpy. 

While the model uses isentropic efficiency, $\eta_s$, 
the efficiency islands on standard compressor maps plot a corrected efficiency, $\eta_{\text{map}}$,
which includes a correction for the Reynolds number at the given operating point.
To convert the corrected map efficiency into isentropic efficiency \parencite{kauffman2006}, 
the transformation
\begin{equation}
  \eta_s = \eta_{\text{map}} + X(1 - \eta_{\text{peak}})\left(1-\left(\frac{10^6}{\RE_b}\right)^{0.1}\right)
  \label{eqn:ReCorr}
\end{equation}
is used. In \cref{eqn:ReCorr}, the flow ratio, $X$, is given by
\begin{equation}
  X = \begin{cases}
        \Theta/\Theta_{\text{peak}} & \text{if } \Theta/\Theta_{\text{peak}} < 1 \\
        1       & \text{if } \Theta/\Theta_{\text{peak}} \geq 1
      \end{cases},
  \nomenclature{$X$}{Flow ratio, $X=\min{(1,\Theta/\Theta_{\text{peak}})}$\nomunit{-}}
  \label{eqn:GetX}
\end{equation} 
where $\Theta_{\text{peak}}$ is the flow coefficient 
corresponding to the point of maximum map efficiency.
Additionally, $\eta_{\text{peak}}$ is the maximum map efficiency
and $\RE_b$ is the Reynolds number 
based on impeller tip width $b$%
\nomenclature{$b$}{Impeller tip width\nomunit{\foot,\,\inch}}; that is,
\begin{equation}
\text{Re}_b=\frac{\rho \vel b}{\mu},
\nomenclature[A$Re$]{$\RE_{(\;)}$}{Reynolds number based on characteristic length $(\;)$, $\text{Re}_{(\;)}=\rho \vel (\;)/\mu$\nomunit{-}}
\nomenclature[W]{$\mu$}{Dynamic viscosity\nomunit{\poundforce\minute\per\foot\squared}}
\end{equation}
with fluid properties evaluated at total suction conditions.

The four map parameters for the desired operating point 
($\Theta$, $\Omega$, $\MA$, and $\eta_{\text{map}}$), 
specified suction conditions ($T_{\inl}$ and $p_{\inl}$), 
and machine characteristics ($D$, $b$, $\Theta_{\text{peak}}$, and $\eta_{\text{peak}}$) 
allow the model to compute the discharge conditions ($T_{\out}$ and $p_{\out}$) 
and mass flow rate $\dot{m}$. 
%A sample compressor map showing the relationship of the machine
%characteristics is shown in \cref{fig:GenMap} on page~\pageref{fig:GenMap}.
%
Once the inlet and outlet states at the compressor are known, 
the mass and energy balances can be revisited. 
From \cref{eqn:SemiMass,eqn:SemiEnergy} and
assuming that the compressor can be modeled as a 
single-input, single-output (SISO) device (no leakage assumption), 
the balance equations become
\begin{align}
  &\text{Mass:}   & 0 &= {\dot{m}_{\inl}} - {\dot{m}_{\out}} \label{eqn:CompMass}\\
  &\text{Energy:} & 0 &= \dot{Q} - \dot{W} + {(\dot{m}h)_{\inl}} - 
		{(\dot{m}h)_{\out}} \label{eqn:CompEnergy}.
\end{align}
\Cref{eqn:CompMass} simply means that the mass flow 
entering the compressor is equal to that leaving the compressor. 
The mass flow rate is known from the 
calculation procedure with the compressor map parameters. 
Finally, assuming that the compressor is well-insulated, 
$\dot{Q}=0$, leaves $\dot{W}$ 
as the sole remaining unknown in \cref{eqn:CompEnergy}. 
This value is one of the required outputs of the model. 

The value calculated for $\dot{W}$ in the compressor is known
as the \emph{gas horsepower} because it represents the power input
directly to the gas. The actual power requirements of the
prime mover are slightly larger as a result of mechanical inefficiencies
in the speed-increasing gearbox and compressor itself.
These losses are modeled assuming the form
\begin{equation}
  \dot{W}_{\text{gas}} = \eta_{\text{mech}}\dot{W}_{\text{mech}},
\end{equation}
with an empirical mechanical efficiency factor $\eta_{\text{mech}}$
which is discussed further in \cref{cha:solnmethod}.

\section{Orifice Flow Meter} \label{sec:Orifice}
\textcite{ptc19} presents correlations relating the orifice differential pressure to the
flow rate for flow measurement purposes.
In the model, the flow rate is known and the differential pressure is to be predicted,
in order to facilitate orifice selection.
The relationships listed in this section hold for flange taps as used
on the \IP{1500}{\horsepower} test block's three flow measurement stations.

First, the volume flow rate based on the conditions 
upstream of the orifice is given by
\begin{equation}
  \dot{V} = \dot{m}_{\inl}v_{\inl},
\end{equation}
which is used elsewhere in the correlations.
The orifice (diameter $d$) and 
pipe (diameter $D_{\text{pipe}}$) cross-sectional areas, 
respectively, are given by
\begin{align}
  A_d &= \frac{\pi}{4} d^2 
  \nomenclature{$d$}{Orifice diameter\nomunit{\foot,\,\inch}} \\
  \shortintertext{and}
  A_D &= \frac{\pi}{4} D_{\text{pipe}}^2.
  \nomenclature{$D$}{Diameter\nomunit{\foot,\,\inch}}
\end{align}
Then  the average velocity, $\vel_D$, in the pipe is
\begin{align}
  \vel_D &= \dot{V}/A_D, \\
  \intertext{and the corresponding Reynolds number is}
  \RE_{D,\text{pipe}} &= \frac{\rho_{\inl} \vel_D D_{\text{pipe}}}{\mu_{\inl}}.
\end{align}
The velocity and Reynolds number at the orifice can be calculated using 
the same equations but substituting $d$ and $A_d$ instead of $D$ and $A_D$.
Even though compressibility effects are present, it is assumed that the pipe inlet properties
can be used for calculations at the orifice location because the effect of compressibility
on thermophysical properties will be small over the differential pressure across the orifice.

For convenience, PTC~19.5 also defines a ratio of diameters, 
$\beta = {d}/{D_{\text{pipe}}}%
\nomenclature[W]{$\beta$}{Ratio of orifice diameter to pipe diameter, $\beta = d/D$\nomunit{-}}$,
which is used throughout the calculations.
For flange taps, the correlations relating differential pressure to mass flow rate take a nested form.
The orifice coefficient of discharge, $C$, is represented by
\begin{align}
C &= K\sqrt{1-\beta^4},
\nomenclature{$C$}{Orifice coefficient of discharge\nomunit{-}}\\
\shortintertext{where the flow coefficient}
K &= K_0 \left(1+\frac{\alpha}{\RE_d}\right).
\nomenclature{$K_{(\;)}$}{Flow coefficients for orifice correlations; refer to \cref{sec:Orifice} for details\nomunit{-}}
\end{align}
Here,
\begin{equation}
\alpha = d\left(\IP{830}{\per\inch} - (\IP{5000}{\per\inch})\beta + (\IP{9000}{\per\inch})\beta^2 - (\IP{4200}{\per\inch})\beta^3 + \frac{\IP{530}{\per\inch^{0.5}}}{\sqrt{D}}\right)
\nomenclature[W]{$\alpha$}{Correction factor for orifice correlations; refer to \cref{sec:Orifice} for details\nomunit{-}}
\end{equation}
and
\begin{equation}
K_0 = K_e \left(\frac{10^6 d}{10^6 d + (\IP{15}{\inch})\alpha}\right).
\end{equation}
The expression for $K_e$ uses singularity function notation; that is,
if the quantity in angle brackets is negative, then that term becomes zero.
\begin{multline}
K_e = 0.5993 + \frac{\IP{0.007}{\inch}}{D} + \left(0.364 + \frac{\IP{0.076}{\inch^{0.5}}}{\sqrt{D}}\right)\beta^4 
      + 0.4\left(1.6 - \frac{\IP{1}{\inch}}{D}\right)^5 \langle 0.07 + {\IP{0.5}{\inch}}/{D} - \beta\rangle^{5/2} \\
      - \left(0.009 + \frac{\IP{0.034}{\inch}}{D}\right) \langle0.5 - \beta\rangle^{3/2} 
      + \left(\frac{\IP{65}{\inch\squared}}{D^2} + 3\right)\langle\beta - 0.7\rangle^{5/2}.
\nomenclature[Y]{$\langle\;\rangle$}{Singularity function notation, $\max{(0,\langle\;\rangle)}$ \nomunit{$\left[\;\right]$}}
\label{eqn:Ke}
\end{multline}

To account for the compressibility of the refrigerant, PTC~19.5 defines
the expansion factor,
\begin{equation}
Y = 1 - (0.41 + 0.35 \beta^4) \frac{\Delta p}{p_{\inl}\gamma},
\nomenclature{$Y$}{Expansion factor\nomunit{-}}
\end{equation}
where $\Delta p$ is the orifice differential pressure, $p_{\inl}-p_{\out}$, and
$\gamma\nomenclature[W]{$\gamma$}{Ratio of specific heats, $\gamma = c_p/c_v$\nomunit{-}}$ 
is the ratio of specific heats evaluated at upstream conditions, 
$\gamma = c_p/c_v\nomenclature{$c_p$, $c_v$}{Specific heats, constant pressure and volume, respectively\nomunit{\btu\per\poundmass\rankine}}$.
\nomenclature{$x$}{Quality\nomunit{-}}%
\nomenclature[X]{$i$}{The $i$th component/element of a mixture/series}%
\nomenclature[Y]{$\Delta\left(\;\right)$}{Change in quantity; for example, $\text{final} - \text{initial or outlet} - \text{inlet}$\nomunit{$\left[\;\right]$}}%
\nomenclature[X]{$\text{sh}$}{Superheat}%
\nomenclature[0]{\textbf{Symbol}}{\textbf{Description}\nomunit{\textbf{Units}}}%
The model also has provisions for choked flow, which may occur when 
running the orifice selection routine over large flow ranges with the smaller orifices.
From \textcite{munson2009}, 
\begin{equation}
p_{\out,\text{choked}} = p_{\inl} \left(\frac{2}{\gamma+1}\right)^{\gamma/(\gamma-1)};
\label{eqn:choked}
\end{equation}
and if $p_{\out,\text{choked}} > p_{\out,\text{free}}$, where free denotes 
the un-choked outlet pressure, then the flow is said to be choked.
The model selects the larger of the two pressures to continue calculations.
Finally, the compressible orifice equation is used to relate the differential pressure to the mass flow rate:
\begin{equation}
\dot{m} = C Y A_d \sqrt{2\rho_{\inl} \Delta p}.
\end{equation}

In addition to the pressure calculations, mass and energy balances are required
for this single-input, single-output device:
\begin{align}
  &\text{Mass:}   & 0 &= {\dot{m}_{\inl}} - {\dot{m}_{\out}} \\ %\to {\dot{m}_{\text{o}}} = {\dot{m}_{\text{i}}} \label{eqn:OrifMass}\\
  &\text{Energy:} & 0 &= \dot{Q} - \dot{W} + {(\dot{m}h)_{\inl}} - 
		{(\dot{m}h)_{\out}} \label{eqn:OrifEnergy}.
\end{align}
The orifice is modeled as a rigid, well-insulated control volume with no shaft work. 
So $\dot{Q}=\dot{W}=0$, and thus $h_{\out} = h_{\inl}$ for the orifice.

\section{Flow Split} \label{sec:FlowSplit}
The flow split is assumed to be rigid and well-insulated with a negligible pressure drop.
Additionally, the two outlet states are modeled as being the same as the inlet state, so
\begin{align}
  T_{\out} &= T_{\out,\text{HGBP}} = T_{\inl} \\
  \shortintertext{and}
  p_{\out} &= p_{\out,\text{HGBP}} = p_{\inl},
%  x_{\out} &= x_{\out,\text{HGBP}} = x_{\inl},  
\end{align}
where subscript ``$\out$'' indicates the main outlet flow stream (leading to the condenser)
and subscript ``$\out,\text{HGBP}$'' indicates the flow stream bypassing the condenser.
The actual division of mass flow rate between the two outlets is determined elsewhere in the model, 
by requiring that the throttled condensed stream mix with the throttled hot gas bypass
stream in the correct proportion to re-establish the specified compressor suction condition.
%So for the flow split, it is sufficient to specify the outlet states equal to the inlet:

\section{Condenser} \label{sec:Condenser}
The condenser has two flow streams: a refrigerant stream and a water stream.
These streams are unmixed, so the mass balances for each stream are given by
\begin{align}
  \dot{m}_{\rfr,\inl} &= \dot{m}_{\rfr,\out} \\% (=\dot{m}_{\rfr})\\
  \shortintertext{and}
  \dot{m}_{\wat,\inl} &= \dot{m}_{\wat,\out}, %(=\dot{m}_{\wat}),
\end{align}
where $\dot{m}_{\rfr}$ and $\dot{m}_{\wat}$ are used to simplify the notation
for the refrigerant and water streams, respectively.

In the condenser, both the water and refrigerant streams are modeled as having constant pressure.
The condenser is modeled as a rigid control volume with no shaft work so that $\dot{W}=0$.
It is also assumed to be well-insulated so that the heat transfer across the external boundary $\dot{Q}=0$.
The internal heat transfer is modeled using the $\NTU$-effectiveness method 
derived by \textcite{incropera2007}; this section contains the pertinent results.

For the condenser in the gas test block refrigeration cycle,
the refrigerant is always the hot fluid and the water is always the cold fluid.
Therefore, the heat capacities for the hot and cold flow streams are
\begin{align}
  C_h &= \dot{m}_\rfr c_{p,\rfr} \\
  \shortintertext{and}
  C_c &= \dot{m}_\wat c_{p,\wat},
\end{align}
where the specific heats are evaluated at the inlet conditions of each flow stream.
Although the specific heats vary slightly with temperature, the impact of assuming
constant specific heats is negligible.
In addition, evaluating specific heats at the inlet temperature greatly 
simplifies the problem since the outlet temperatures are initially unknown.
Then $C_{\min} = \min{(C_h,C_c)}$ and $C_{\max} = \max{(C_h,C_c)}$. The $\NTU$
for the condenser is defined by
\begin{equation}
  \NTU = \frac{UA}{C_{\min}},
  \nomenclature[A$N$]{$\NTU$}{Number of transfer units \nomunit{-}}
\end{equation}
where $UA$ is the overall heat transfer coefficient determined from experimental data.

In the $\NTU$-effectiveness method, the heat transfer is modeled in terms of
the effectiveness, $\epsilon$, and the theoretical maximum heat transfer rate,
$\dot{Q}_{\max}$,
which represents the performance of heat exchanger with infinite surface area.
The effectiveness is used to determine the actual heat transfer rate
$\dot{Q}$, which is given by
\begin{equation}
  \dot{Q} = \epsilon \dot{Q}_{\max}.
\end{equation}
For a phase-change application such as the condensation process, 
the effectiveness $\epsilon$ is defined as
\begin{equation}
  \epsilon = 1 - \mathrm{e}^{-\NTU},
\end{equation}
while the maximum theoretical heat transfer rate $\dot{Q}_{\max}$ is 
calculated using
\begin{equation}
  \dot{Q}_{\max} = \dot{m}_\rfr (h_{\rfr,\inl} - h_{\rfr,T=T_{\wat,\inl}}),
\end{equation}
which represents fully condensing and cooling the refrigerant stream to the
inlet water temperature. 

Finally, the enthalpies of the water and refrigerant exiting the condenser
can be determined from an energy balance for each flow stream:
\begin{align}
  0 &= -\dot{Q} + \dot{m}_{\rfr}(h_{\rfr,\inl} - h_{\rfr,\out}) \\
  \shortintertext{and}
  0 &= \dot{Q} + \dot{m}_{\wat}(h_{\wat,\inl} - h_{\wat,\out}).
\end{align}
Any other properties of the water and refrigerant exiting the condenser,
such as temperatures, can be determined based on the known pressures and enthalpies.

\section{Cooling Tower} \label{sec:CoolingTower}
The JCI gas test block has multiple rooftop cooling towers to reject
energy from the cycle to the outdoor air.
To simplify the model, these physical cooling towers are lumped into
a single component.
Additionally, experimental data for validation are only available
on an overall basis and not for each individual tower.
The aggregated cooling tower has two primary flow streams: a water stream
requiring cooling and outdoor air, which is treated as a mixture of air and water vapor.
These flow streams mix in the cooling tower and a portion of the water stream
evaporates into the air-water vapor mixture.
By convention, the mass balance for the air-water vapor mixture
is written in terms of the dry air mass flow rates, which are equal 
for a steady-flow system:
\begin{equation}
  \dot{m}_{\air,\inl} = \dot{m}_{\air,\out}.
\end{equation}
As in the condenser model, the notation $\dot{m}_{\air}$ will be used throughout
for simplicity.
For a mass balance on the water stream, an additional term, $\dot{m}_{\wat,\text{evap}}$, accounts for
the water evaporated into the moist air stream,
so the mass balance is
\begin{equation}
  \dot{m}_{\wat,\inl} = \dot{m}_{\wat,\out} + \dot{m}_{\wat,\text{evap}},
\end{equation}
where the water lost to evaporation is given by
\begin{equation}
  \dot{m}_{\wat,\text{evap}} = \dot{m}_{\air} (\omega_{\air,\out} - \omega_{\air,\inl}).
  \nomenclature[W]{$\omega$}{Absolute humidity ratio \nomunit{\poundmass_{\wat}\per\poundmass_{\air}}}
  \nomenclature[X]{$\wat$}{Water}
  \nomenclature[X]{$\air$}{Air-water vapor mixture}
  \nomenclature[X]{$\rfr$}{Refrigerant}
  \label{eqn:watevap}
\end{equation}
In \cref{eqn:watevap}, $\omega_{\air,\out}$ and $\omega_{\air,\inl}$ 
represent the absolute humidity ratio of the 
air-water vapor mixture at the cooling tower outlet and inlet, respectively.

\textcite{braun1989} developed correlations which are used to apply
the $\NTU$-effectiveness method to the cooling tower. First, the
saturation specific heat, $c_{\sat}$, is approximated by
\begin{equation}
  c_{\sat} = \frac{h_{\air,\sat,\inl}-h_{\air,\sat,\out}}{T_{\wat,\inl} - T_{\wat,\out}},
  \nomenclature{$c_{\sat}$}{Saturation specific heat \nomunit{\btu\per\poundmass\rankine}}
  \nomenclature[X]{$\sat$}{Saturated}
\end{equation}
where subscript $\sat$ indicates that the enthalpy is to be evaluated 
for a saturated air-water vapor mixture.
An effective mass flow rate, $m^*$, is defined as
\begin{equation}
  m^* = \frac{\dot{m}_{\air}}{\dot{m}_{\wat,\inl} c_{p,\wat}/c_s},
\end{equation}
where the water specific heat, $c_{p,\wat}$, is evaluated at the inlet conditions.
The number of transfer units ($\NTU$) for the cooling tower is calculated using
a semi-empirical approach:
\begin{equation}
  \NTU = c \bigg( \frac{\dot{m}_{\wat,\inl}}{\dot{m}_{\air,\inl}} \bigg)^{1+n},
\end{equation}
where $c$ and $n$ are empirically determined constants for the cooling tower.
The cooling tower effectiveness, $\epsilon_{\air}$, is then defined as
\begin{equation}
  \epsilon_{\air} = \frac{1-\exp{\big(-\NTU(1-m^*)\big)}}{1-m^*\exp{\big(-\NTU(1-m^*)\big)}}.
  \nomenclature[W]{$\epsilon$}{Heat exchanger (or cooling tower) effectiveness, $\dot{Q}/\dot{Q}_{\max}$ \nomunit{-}}
\end{equation}

Analogous to the energy balances for the condenser, the energy balance for the
air-water vapor stream is
\begin{equation}
  \dot{Q} = \epsilon_{\air} \dot{Q}_{\max},
\end{equation}
where the maximum possible heat transfer, $\dot{Q}_{\max}$, is calculated 
by assuming that the outlet air is fully saturated at the inlet water temperature:
\begin{equation}
  \dot{Q}_{\max} = \dot{m}_{\air} (h_{\air,\sat,T=T_{\wat,\inl}} - h_{\air,\inl}).
\end{equation}
The outlet air state is then calculated using
\begin{equation}
  \dot{Q} = \dot{m}_{\air} (h_{\air,\out} - h_{\air,\inl}).
\end{equation}
Finally, the outlet water temperature is given by
\begin{equation}
  T_{\wat,\out} = T_{\wat,\inl} - \frac{\dot{m}_{\air,\inl}(h_{\air,\out} - h_{\air,\inl})}{\dot{m}_{\wat,\inl} c_{p,\wat}}.
\end{equation}

\section{Expansion Devices} \label{sec:Expand}
The actual gas test block cycle has a complicated arrangement 
of numerous valves and spray nozzles which simultaneously throttle and mix 
the condensed refrigerant stream with the bypassed refrigerant stream 
to reestablish the compressor suction conditions.
For the purposes of a one-dimensional model, this complex arrangement
may be separated into three distinct processes: 
throttling the condensed refrigerant stream to the cycle's low pressure (suction pressure),
throttling the hot gas bypass refrigerant stream to the cycle's low pressure, and
mixing the two refrigerant streams in the correct proportions 
to achieve the specified test suction conditions.

While the condensed refrigerant stream is a subcooled liquid 
or saturated liquid-vapor mixture and the hot gas bypass refrigerant stream
is a superheated vapor, the theory for the throttling process is the same
in both cases, as described in this section.
The theory for the mixing process is described in \cref{sec:Mixing}.

The throttle is modeled as a rigid control volume with no shaft work ($\dot{W}=0$).
Additionally, it is assumed to be well-insulated ($\dot{Q}=0$). The mass balance for this single input,
single output system is
\begin{equation}
  \dot{m}_{\inl} = \dot{m}_{\out}.
\end{equation}
The energy balance then reduces to
\begin{equation}
  h_{\inl} = h_{\out}.
\end{equation}
The pressure at the exit of each throttle is assumed to be the 
compressor suction pressure, and thus the outlet states of the
throttling devices are fixed by the known enthalpy and pressure.

\section{Mixing Chamber} \label{sec:Mixing}
The mixing chamber also is assumed to be a rigid control volume which is well-insulated.
It has two input streams (condensed and bypassed refrigerant) and one outlet leading to the
compressor suction inlet.
The mass balance for the mixing chamber is
\begin{equation}
  \dot{m}_{\inl} + \dot{m}_{\inl,\text{HGBP}} = \dot{m}_{\out}.
\end{equation}
The energy balance is given by
\begin{equation}
  \dot{m}_{\inl}h_{\inl} + \dot{m}_{\inl,\text{HGBP}}h_{\inl,\text{HGBP}} = \dot{m}_{\out}h_{\out}.
\end{equation}
It is assumed that the mixing process occurs at a constant pressure.
Therefore, the two balances on the mixing chamber constrain the division of flow between
the condensed and bypassed streams by requiring that their mixing re-establishes
the specified suction condition.
