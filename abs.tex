\begin{abstract}
The increasingly competitive building equipment and control industry 
pushes manufacturers to continually improve the performance and efficiency 
of their products to develop and maintain a competitive edge.  
Compressor development is an expensive endeavor, but the cost and time required for testing can be 
minimized by developing a model of the compressor test block to predict 
its behavior with a given prototype compressor at specified operating conditions.
This thesis presents a thermodynamic model of a hot gas bypass test block 
used to evaluate centrifugal compressor performance at a compressor development facility.  

The test block uses cooling towers to 
reject the heat of compression to outdoor air, and experience has shown that 
the range of achievable compressor test conditions can be limited by 
outdoor air temperature and humidity, which affect the heat transfer rate.  
Therefore, one goal of the model development was to provide 
a means for evaluating the feasibility of tests at given outdoor air conditions.  
By incorporating models of the cooling towers into the test block model, 
test engineers now are able to predict the range of compressor 
suction and discharge conditions that can be tested under the 
current outdoor air conditions.  

A second goal of the model was to assist in selecting the orifice plate 
used in the orifice flow meter that measures mass flow through the compressor.  
Engineers previously had to make an educated guess as to the best 
orifice plate size in advance of running the tests, but the model now 
identifies the orifice diameters that result in differential pressures 
within the desired range, minimizing the trial and error involved in testing.

The model assumes that the system operates at steady-state conditions 
and uses a compressor map to model expected prototype compressor performance.  
Therefore, this research focuses on the condenser and cooling tower models, 
which are the most important elements for predicting the impact of 
outdoor conditions on cycle performance.  
It is shown that the resulting model achieves agreement to within 
2.5\% of experimental data. 
The results for orifice differential pressure agree to within 
0.35\% of experimental data, providing a useful orifice selection routine.
\end{abstract}