\chapter{Conclusions and Recommendations} \label{cha:conclusion}
This chapter contains a summary of the research work presented in this thesis.
Conclusions from the results are presented and discussed.
Finally, future improvements to the model
and test block data acquisition are recommended.
These data acquisition improvements could expand the areas of the model that can
be validated, which in turn will improve the accuracy of the model by way of
better parameter selection.

The motivations and requirements in \cref{cha:intro} guided the development
of a thermodynamic model of the \IP{1500}{\horsepower} hot gas bypass test block
facility located in York, PA.
The simplifying assumptions and prior research (\cref{cha:litreview}) 
used in developing the model were outlined in \cref{cha:modeldev}.
These simplified equations are numerically solved by an implementation
in EES as discussed in \cref{cha:solnmethod}.
Graphical interfaces are presented for the two main use cases of the model:
the complete cycle simulation and the orifice selection program.

\Cref{cha:results} shows the results of a validation effort,
in which model outputs are compared to experimental data from the test block.
These results show that the model is accurate enough for the intended application
and that the assumptions applied in developing the equations are appropriate.
Sensitivity analyses conducted on various inputs and calculations show that,
even when intermediate results do not fully agree with experimental data,
the primary outputs remain accurate enough.

\section{Conclusions} \label{sec:conclusions}
Based on the results presented and discussed in \cref{cha:results},
the model meets the goals of predicting the air mass flow rate in the
cooling tower and predicting the orifice differential pressure.
If the predicted air mass flow rate exceeds the capacity of the cooling towers,
then it is likely that the specified test condition is infeasible at the given
outdoor air conditions.
For the orifice selection program, engineers are able to specify
a desired range of test conditions.
The model will then display the orifice differential pressure for 
a number of operating points in the specified range, allowing
the engineer to make an informed decision about the best orifice 
for the test plan without tedious calculations.

For the validation cases of \cref{cha:results}, the maximum percent difference between model
and experimental results is 2.5\%, so the model is quite accurate given the number of assumptions involved.
The primary output variable for orifice selection, the orifice differential pressure,
agrees with the experimental results to within 0.35\%.
It should be noted, however, that the validation encompassed only a
small range of test points and weather conditions, from which several
of the parameters in the model are defined.
Results for operating conditions far removed from the validation data
may vary.
Methods for correcting this shortcoming of the model are discussed in \cref{sec:recommend}.

\section{Recommendations} \label{sec:recommend}
As noted in \cref{cha:results} and \cref{sec:conclusions},
the condenser and cooling tower modules of the model make significant
assumptions concerning parameters.
These are the overall heat transfer coefficient for the condenser, $UA$,
and the two characteristic parameters for the cooling tower model's 
calculation of $\NTU$, $c$ and $n$.
These parameters are affected by many complex transport processes,
and first-principles calculation of these was not a central focus of this project.
Instead, these parameters were defined empirically, using averages over the 
validation data sets.
For this reason, the model results may not be as accurate for test conditions
far removed from the validation test conditions.
To improve these portions of the model, a number of data collection
recommendations are prudent.

First, the cooling tower fan speeds should be recorded as part of each test.
Currently, the fan speeds are adjusted until the test block reaches steady state
at the desired operating point.
Combined with fan curves for the cooling towers installed on the block,
records of the fan speeds will allow calculation of the cooling tower
model parameters $c$ and $n$. 
This will enhance the accuracy of the cooling tower and extend the range
of test conditions over which its accuracy is acceptable.
At the present time, the model uses typical values for these parameters from \textcite{braun1989}.

Similarly, a means to determine the condenser water flow rate would
greatly improve calculation of the condenser overall heat transfer coefficient, $UA$.
As discussed in \cref{sec:sensitivity}, the condenser water pump drives the
flow in both a bypass loop and the condenser loop. With the current instrumentation
setup, the division of flow between these loops is unknown.
If the condenser water flow rate is known, then a more reliable
estimate of $UA$ could be produced.

Other potential improvements to the model concern the adiabatic and constant pressure assumptions.
It is shown in \cref{sec:condresult} that the constant pressure assumption is valid for the sizes
of pipe used on the gas test block.
Additionally, the adiabatic assumption is shown to be valid for the orifice in \cref{sec:OrifResults}.
However, the adiabatic assumption for the compressor results in underprediction of the compressor
gas horsepower, as discussed in \cref{sec:CompResults}.
A model of the heat transfer from the compressor casing to the ambient conditions is likely to
improve this prediction.
Only the compressor module and the overall energy balance 
for the cycle would need to be modified to implement this improvement.

\emph{How to end? Already summarized everything at the beginning.}