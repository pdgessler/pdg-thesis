\chapter{Literature Review} \label{cha:litreview}
As discussed in \cref{cha:intro}, thermodynamic models of each component
in the hot gas bypass test cycle have already been created and are widely used.
This literature review surveys the existing test methods and models 
and summarizes information about the JCI
test block cycle, including equipment details. A general description of each
source is contained here, while any mathematical models used are presented
(with citations) in the complete model development of \cref{cha:modeldev}.

\section{General Compressor Testing} \label{sec:gencomp}
One component of the research effort was 
to study different methods of testing compressors. 
Therefore, this section provides information regarding compressor performance metrics 
and testing methods commonly used in the HVAC industry. 
The research was done to better understand the testing methodology, 
the data collected during a typical test, 
and the expectations for a mathematical model of the test block refrigeration cycle.

\subsection{Compressor Testing Methodology---ASME PTC 10} \label{sec:TestMethodology}
The AMSE PTC 10-1972 standard \parencite{ptc10} prescribes test conditions, 
procedures, and measurement locations for a compressor test. 
This allows different manufacturers' test results to be compared. 
The standard also presents dimensionless coefficients used by different 
industries to characterize the operating point of the compressor
with associated conversions. 
The relevant coefficients are described in greater detail as part of the 
compressor model development in \cref{sec:Compressor}.
These coefficients serve as inputs to the thermodynamic model
that fully characterize the performance of the compressor.
Because the compressor operating point constrains
many variables in the model, these coefficients 
are critical to the model's performance.

Significant portions of PTC 10 focus on multistage compressor testing
and sideload calculations. These are not needed for the present
modeling effort, since the vast majority of recent compressor tests
conducted on the \IP{1500}{\horsepower} gas test block use 
single-stage compressor setups \parencite{trevino2012}.

Finally, ASME PTC 10 contains several sets of sample calculations 
which are of use in verifying the implementation of the model, 
particularly the calculation of the prime mover power requirements.
These calculations are similar in form to those used in the existing
compressor test block data acquisition program and the compressor
module of the present thermodynamic model.

\subsection{Compressor Performance Metrics}
It is important to understand the specific quantities used 
to describe a compressor's performance within the HVAC\&R%
\nomenclature[Z]{HVAC}{Heating, Ventilating, and Air Conditioning} 
industry, since these will be among the outputs of any model of the testing process. 
\textcite{wilcox2007} advocates the use of ASME PTC-10 (\cref{sec:TestMethodology}) and lists
%\begin{itemize}
  suction and discharge pressures,
  suction and discharge temperatures,
  mass flow rate,
  fluid (refrigerant) composition,
  rotational speed, and
  driver load
%\end{itemize}
as critical field data for any compressor test.
These critical field data also appear as inputs or outputs 
of the thermodynamic model of the compressor, presented in \cref{sec:Compressor}.

Furthermore, Wilcox outlines some general guidelines 
for instrumentation on the test block to ensure representative data.  
He notes that pressure and temperature sensors should be located 
at least 10 pipe diameters from potential disturbances or obstructions such as tees or elbows.  
All sensors should be calibrated prior to the test run, 
and data should only be collected at steady-state conditions.  
For a typical compressor test, Wilcox defines steady-state conditions to be achieved 
once the discharge temperature remains constant (within sensing accuracy) over a \SI{15}{\minute} interval. 
According to Wilcox, resistance temperature detectors (RTDs)%
\nomenclature[Z]{RTD}{Resistance Temperature Detectors} 
should be used instead of thermocouples wherever possible for improved accuracy. 
Finally, he stresses the importance of recording test data at several different operating points 
to allow recognition of a bad measurement in any one set of data.
All of these guidelines are met by the instrumentation and testing procedures
in use on the \IP{1500}{\horsepower} gas test block at JCI.
This ensures that validation data provide a good representation
of the actual operation of the system.

%\subsection{Compressor Test Data Analysis}
%\todo{Summarize Wilcox (and others) compressor performance calculations\dots}

\section{Existing Component Thermodynamic Models}
The \IP{1500}{\horsepower} gas test block contains a number of components which must be modeled. 
The compressor is not directly modeled in this work, 
which instead uses external compressor maps or other means 
to specify the discharge conditions given a set of suction conditions
as discussed in \cref{sec:TestMethodology,sec:Compressor}. 
The flow measurement orifice, hot gas bypass (HGBP) flow split, condenser, cooling towers, 
throttling expansion devices, and a mixing chamber must be modeled. 
This section summarizes a number of texts and papers that were consulted to model these devices.

\subsection{Flow Measurement Orifice}
The flow measurement orifice is modeled according to correlations presented in 
ASME PTC 19.5 \parencite{ptc19} and following the general form of \textcite{munson2009}. 
The ASME PTC 19.5 standard presents relationships between 
the pressure drop across the orifice and the flow rate, as shown in \cref{sec:Orifice}. 
Compressible effects are included since the refrigerant will be in a 
superheated vapor state at the orifice.
According to \textcite{trevino2012} and \textcite{flow1984}, flange taps
are used on the test block, so the correlations in ASME PTC 19.5
for flange tap pressure measurements are used.

\subsection{Condenser}
\textcites{incropera2007}{incropera1985}, and \textcite{kays1984} 
developed equations describing the performance of the condenser, 
which in this case is a shell-and-tube unit. 
The $\NTU$-effectiveness method is used to calculate the heat transfer rate
in terms of inlet and outlet temperatures since internal temperature measurements are not available. 
Engineering drawings from JCI \parencite{cond1983} and a technical manual
for the condenser \parencite{condman} were used, along with experimental data,
to determine the effectiveness and number of transfer units for the condenser.

\subsection{Cooling Tower}
\textcite{braun1989} and \textcite{mitchell2013} developed equations for modeling cooling towers 
using an effectiveness approach analogous to the heat exchanger model.
This is useful as it avoids iteration wherever possible
and does not require detailed cooling tower data. 
An example useful for verification of the cooling tower model 
is presented by \textcite{mitchell2013}. 
The model requires two characteristic performance parameters, 
which affect the number of transfer units ($\NTU$) of the cooling tower.
Therefore, a linear regression process is performed on experimental data from 
the Baltimore Aircoil Company\nomenclature[Z]{BAC}{Baltimore Aircoil Company} 
(BAC) cooling towers \parencite{CLTselect} 
to determine these parameters.
\textcite{braun1989} outlined this process and provide several values 
for typical cooling towers. 
\textcite{qureshi2006} extended the model to include fill fouling.
The effect of fill fouling is not included currently, but could be added 
in the future if desired.

\subsection{Minor Components}
The flow split, expansion devices, and mixing chamber are modeled using 
basic extensive property balances (mass and energy) as detailed in 
undergraduate thermodynamics texts such as \textcite{cengel2011}. 
The equations are developed from the general balance equations 
in \cref{sec:FlowSplit,sec:Expand,sec:Mixing}.

\section{Test Block Design and Construction}
In addition to the York documentation already cited, further details
regarding the test facility are found in the Applied Systems Operating Limits
report \parencite{bluebook}. This includes pipe diameters, orifice diameters, and
operating instructions. The condenser water pump is a fixed-speed unit,
Aurora Type 410, size $6\times8\times15$, with pump curves given by
\textcite{aurora2007}.

The US Patent by Sahs and Mould~\parencite{sahs1956}, \textcite{dirlea1996},
and \textcite{mcgovern1984} describe the general configuration
of the cycle, and illustrate how the different components interact with one 
another.
However, no holistic modeling effort is attempted in any of these sources.

\section{Summary}
To the best of the author's knowledge, a complete model of the hot gas bypass
test block cycle is a novel undertaking.
The completion of this model will have significant impact on the testing process
used every day by test engineers at JCI \parencite{sommer2013}.
The survey of literature contained in \cref{cha:litreview} provides
the theoretical basis for each component-level model in the complete cycle.
\cref{cha:modeldev} will discuss each component model in greater detail.
%Existing descriptions of the hot gas bypass test cycle may be found in the
%literature, but, to date, a complete thermodynamic model of the cycle has
%not been developed.
%The construction of such a model will assist test engineers in their daily
%work and reduce testing costs and time.