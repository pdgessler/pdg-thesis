\chapter{Introduction} \label{cha:intro}
The increasingly competitive building equipment and control industry 
pushes manufacturers to devote more resources each year to research and development, 
continually improving the performance and efficiency of their products to develop a competitive edge. 
The design of centrifugal compressors is no exception to this trend, 
since these compressors are used in many power-intensive applications
including water-cooled chillers in large commercial and industrial buildings. 
The compressor transfers energy from the source (often a shaft driven by an electric motor) 
into the refrigeration cycle, and is thus a scrutinized component of the design.

Compressor development is an expensive endeavor because of prototyping and testing costs. 
The design of a new compressor requires at least one prototype to be constructed, 
which is then subjected to a series of industry-standard performance tests 
to quantify the improvements of the new design.
This adds cost and development time to an already expensive process. 
For example, the costs for one week of testing can approach \SI{10000}[\$]{} \citep{sommer2013}.
The requirement of physical performance tests should not be disputed, 
since the resulting ratings are used by potential customers to compare offerings from different manufacturers. 
However, minimizing testing time has a large impact on costs and time to market. 
A model based on first principles has the potential to reduce the testing time 
and thus decrease costs by assisting the test engineer in defining an optimized test plan
built around test block capabilities at the expected ambient conditions.
Furthermore, we can minimize down time for changing flow measurement orifices
by providing a means to choose the best orifice diameter for a given range of test flow rates.

The overall goal of this research is to 
create a numerical model to simulate the \IP{1500}{\horsepower} gas block 
compressor testing equipment used by Johnson Controls, Inc.~%
(JCI)\nomenclature[Z]{JCI}{Johnson Controls, Inc.} 
at their facility in York, Pennsylvania. 
The model will use the full-load design conditions of the new compressor (flow, head, and efficiency) 
to determine the test setup (flow measurement orifice size and cooling tower fan speed) 
required for given ambient conditions. 
The current testing process requires some trial and error to find a suitable setup for a new compressor. 
The numerical model aims to quickly provide reasonably accurate initial estimates of the orifice diameter and cooling tower fan speed 
required to test a new compressor at specified outdoor air conditions, 
reducing the testing time by eliminating (or at least minimizing) the trial and error phase of the testing process.

\section{Background} \label{sec:background}
A preferred method of evaluating compressor performance is to use a closed-loop gas test block
using the design process fluid (refrigerant) at design flow conditions. 
While the equipment can be expensive to construct, operate, and maintain, 
the gas test block makes isolating the compression portion of the refrigeration cycle easier. 
The basic premise of the gas test block cycle is that 
the cycle can maintain conditions at the compressor inlet similar to those experienced in a traditional vapor-compression refrigeration cycle, 
while the conditions at other points in the cycle need not follow the traditional refrigeration cycle arrangement. 
Temperature-entropy (\tssymb) 
\nomenclature{$T$}{Temperature\nomunit{\kelvin}} \nomenclature{$s$}{Specific entropy\nomunit{\kilo\joule\per\kilogram\kelvin}}
diagrams for the typical vapor-compression refrigeration cycle 
and an idealized gas block test cycle are shown in \cref{fig:CycleComparison}. 
\begin{figure}[htbp]
  \centering
  \subbottom[Vapor-compression refrigeration cycle.]{%
		\includegraphics{figs/BasicChiller}%
    \label{fig:CycleComparisonChiller}%
	}
  \hfill
  \subbottom[Gas test block cycle.]{%
		\includegraphics{figs/BasicGas}%
    \label{fig:CycleComparisonGasBlock}%
	}
  \caption{Comparison of idealized temperature-entropy (\tssymb) diagrams.}
  \label{fig:CycleComparison}
\end{figure}

There are energy savings associated with using the gas test block 
instead of the chiller refrigeration cycle. 
\Process{4}{1} of the traditional refrigeration cycle of \cref{fig:CycleComparisonChiller} 
is the result of heat transfer into the evaporator, 
which is the building cooling load or refrigeration effect. 
In a chiller test block situation, this load is simulated 
by mixing water from the condenser and evaporator loops. 
In the gas block test cycle, no water loop is required for this process, 
\state{1} being reestablished instead by mixing saturated liquid-vapor mixture (\state{4}) 
with a superheated vapor separated from the discharge stream and throttled to the cycle's low pressure (\state{5}). 
The conditions at \state{1} are controlled by adjusting 
the discharge flow split that occurs near \state{2} of \cref{fig:CycleComparisonGasBlock}. 
In addition to eliminating one water loop from the system, the condenser heat transfer%
---\process{2}{3} of \cref{fig:CycleComparisonGasBlock}---%
is also reduced, since only a portion of the refrigerant flow must be condensed.

A primary benefit of this arrangement is that the gas block is 
more versatile than a traditional refrigeration cycle used in a chiller. 
The gas block can handle a wide variety of test gases (refrigerants) 
and their associated operating pressures and cooling loads, 
while a chiller may require different heat exchangers or piping 
to operate with certain refrigerants at a full range of operating conditions.
Additionally, the gas test block provides better locations for instrumentation 
and conforms to industry-standard test codes (\eg{} ASME%
\nomenclature[Z]{ASME}{American Society of Mechanical Engineers} 
PTC\nomenclature[Z]{PTC}{Performance Test Code}-10) 
outlining proven and well-established data analysis and results reporting methods. 
For example, the test code specifies straight sections of suction pipe and/or flow straighteners 
to produce near-axial flow, while the compact piping arrangement on a chiller causes large deviations from axial flow. 
Using a gas test block provides an even basis of comparison for compressors independent of the chiller design. 
A simplified schematic of the test block layout is shown in \cref{fig:SimpSchematic}, 
with state numbering corresponding to \cref{fig:CycleComparisonGasBlock}. 
\begin{figure}[htbp]
	\centering
	\includegraphics{figs/CompressorBlockDiagram}
	\caption{Simplified schematic of the \SI{1500}{hp} gas test block facility.}
	\label{fig:SimpSchematic}
\end{figure}

\section{Motivation} \label{sec:motivation}
As discussed in the opening of \cref{cha:intro}, 
the numerical model of the test block will reduce the time required during physical testing of new compressor designs 
by giving engineers a starting point for the test setup for a particular compressor. 
This setup includes the flow measurement orifice diameter, the mixing valve positions, and the cooling tower fan speeds.

Additionally, the model will check if a set of test conditions 
can be achieved at specified outdoor air conditions. 
These limit the performance of the cooling towers and may preclude certain compressor tests. 
This prediction could prevent lost time and resources expended setting up a compressor test 
for conditions that are not feasible at the current outside temperature and humidity.

Finally, JCI has expressed interest in building an automated test block in the future. 
If desired, the present model could be adapted for use in a model-based controls design workflow 
to expedite and enhance the control system design process. 
This is a long-term motivation and is secondary to the test time reduction and limiting conditions motivations.

\section{Objectives} \label{sec:objective}
The overall goal of this research is to develop a one-dimensional, steady-flow numerical model 
representing the \IP{1500}{\horsepower} gas test block at JCI's York, PA facility. 
The numerical model will be used in conjunction with compressor maps and/or computational fluid dynamics (CFD)%
\nomenclature[Z]{CFD}{Computational Fluid Dynamics}
models of the compressors to quantitatively predict the performance of new compressor designs on the test block. 
To accomplish the overall goal, the model must
\begin{itemize}
  \item compute the power requirements for the prime mover of the test block,
  \item choose the orifice diameter that results in the smallest error in measured
    flow rate over the widest range of operating conditions, and \todo{Need to further define orifice selection.}
  \item report the limits of the test block operating conditions for given compressor 
		design, test block control settings, and ambient air conditions.
\end{itemize}

\section{Requirements and Constraints} \label{sec:requirements}
The numerical model must be easy for engineers in the compressor engineering group at JCI 
to use and update and should minimize dependencies on licensed software for better portability. 
The software tools generally available to the compressor engineering group are 
Engineering Equation Solver (EES)\nomenclature[Z]{EES}{Engineering Equation Solver} and \MLS{}. 
Engineering Equation Solver was chosen since \MLS{} is more suited to dynamic models, 
the engineers at York are more familiar with EES, 
and EES has built-in thermophysical property relations 
while \MLS{} requires interfacing with an external library.

The compressor block model should be one-dimensional and steady-flow in nature; 
that is, the flow conditions at a cross section of flow 
are treated as spatially uniform and constant over time. 
Modeling is conducted on a macroscopic level, 
neglecting the effects of property gradients within the flow, 
such as viscous and thermal boundary layers. 
Modeling such detailed phenomena would complicate the model 
and would not significantly improve the predictions requested by JCI listed in \cref{sec:objective}.

The execution time of the model is a primary concern of JCI's compressor engineers, 
and should be less than \SI{30}{\second} for each individual compressor test.
\todo{Add additional quantitative metrics (percent error for different quantities, interface features, etc.) to this paragraph. Maybe a table if things expand significantly.}

\section{Organization} \label{sec:organization}
This thesis documents the creation of a one-dimensional, steady-flow model 
of a specific gas test block for use by compressor development engineers at JCI's York, PA location. 
\Cref{cha:litreview} summarizes a review of the current state of the art 
compressor testing and modeling techniques 
and other reference materials used in developing and implementing the model. 
\Cref{cha:modeldev} presents the mathematical formulation of the model 
and its inherent engineering assumptions. 
\Cref{cha:solnmethod} documents the implementation of the numerical model, 
including numerical solution techniques and user interface considerations. 
\note{Might combine Chs. 3 and 4.}

\Cref{cha:results} presents the predictions of the model 
for a variety of compressor designs and operating points. 
These predictions are compared to corresponding experimental data 
from physical tests to validate the model predictions.
A discussion of the validation results is included, 
highlighting the strengths and weaknesses of the numerical model.
\Cref{cha:conclusion} summarizes the research work 
and provides recommendations to users of the model and future maintainers.

%\Cref{app:code} contains code listings and screenshots of the EES implementation
%for the compressor gas test block model. 
\todo{Describe any appendices as they are added.}
